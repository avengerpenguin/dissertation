\documentclass{sig-alternate-05-2015}

\begin{document}

\title{Combining Linked data and Data Mining Techniques to improve Clustering of Large Scale Media Repositories: A case study with BBC}
%\subtitle{[Extended Abstract]

\numberofauthors{3}
\author{
\alignauthor
Ross Fenning\\
       \affaddr{British Broadcasting Corporation (BBC)}\\
       \affaddr{MediaCityUK}\\
       \affaddr{Salford, United Kingdom}\\
       \email{ross.fenning@bbc.co.uk}
\alignauthor
Dhavalkumar Thakker\\
       \affaddr{University of Bradford}\\
       \affaddr{Bradford}\\
       \affaddr{United Kingdom}\\
       \email{dhavalkumar.thakker@gmail.com}
\alignauthor Tejal Shah\\
       \affaddr{UNSW}\\
       \affaddr{Australia}\\
       \email{tejals@cse.unsw.edu.au}
}

\maketitle
\begin{abstract}
Media companies produce ever larger numbers of articles, videos, podcasts,
games, commonly collectively known as ``content''. A successful
content-producing organisation not only has to develop systems to aid producing
and publishing content, but there are also demands to engineer effective
mechanisms to aid consumers in finding that content. Linked Data technologies
provide data enrichment beyond attributes and keywords explicitly available
within content data or metadata. In our work we experiment with a
plausible mapping of an RDF graph representing a content item to an attribute
set suitable for data mining. With such a mapping, we have explored the
application of machine learning, particularly unsupervised learning, across
an organisation's whole content corpus. We have created an innovative pipeline
of linked data and data mining that allows clustering of media content items on
the fly. We have evaluated such system in the context of website content
produced by the British Broadcasting Corporation (BBC) and present the optimum
combination of RDF graphs and data mining with qualitative and quantitative
results.
\end{abstract}

\keywords{Semantics; Linked Data; Semantic Web; Machine Learning; Clustering; Data Mining; RDF Graph; Media repositories; Content Management}

\section{Introduction}

Media companies produce ever larger numbers of articles, videos, podcasts,
games, etc. -- commonly collectively known as ``content''. A successful
content-producing website not only has to develop systems to aid producing and
publishing that content, but there are also demands to engineer effective
mechanisms to aid consumers in finding that content.

\subsection{Problem}

Approaches to aid discovery of content used in industry include providing a
text-based search, hierarchical
categorisation (and thus navigation thereof) and even more tailored recommended
content based on past behaviour or content enjoyed by friends (or sometimes
simply other consumers who share your preferences), but to build systems
that operate across the full corpus of their content, organisations face
several problems:

\begin{itemize}

\item Large organisations can have content across multiple content management
systems, in differing formats and data models.

\item Many content items are in fairly opaque formats, e.g. video content may be
stored as audio-visual binary data with minimal, textual metadata to display on
a containing web page.

\item Content is being published continuously, which means any search or
discovery system needs to keep up with content as it is published and process it
into the appropriate data structures. Any analytical process that operates
over all content (e.g. machine learning) may need to be run periodically or in
an incremental fashion.

\end{itemize}

\subsection{Hypothesis}

The following hypotheses are proposed for gaining new insights about an
organisation's diverse corpus of content:

\begin{itemize}

\item Research and software tools around the concept of \emph{Linked Data} can
aid us in rapidly acquiring a broad view (perhaps at the expense of depth) of an
organisation's content whilst also providing a platform for simple enrichment of
that content's metadata.

\item We can establish at least a na\"ive mapping of an RDF graph representing a
content item to an attribute set suitable for data mining. With such a mapping,
we can explore applying machine learning across an organisation's whole content
corpus.

\item Linked Data and Semantic Web \emph{ontologies} and models available can
provide data enrichment beyond attributes and keywords explicitly avaiable
within content data or metadata.

\item Many content-producers currently enrich their web pages with small
amounts of semantic metadata to provide better presentation of that content
as it is shared on social media. This enables simple collection of a full
breadth of content with significantly less effort than direct integration
with content management systems (i.e. Enterprise Integration).

\end{itemize}

\section{Background}

\section{Design}

\section{Implementation}

\section{Analysis}

\section{Evaluation}

\section{Conclusions}

%\bibliographystyle{abbrv}
%\bibliography{../report/references}

\end{document}
