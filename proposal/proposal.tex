\documentclass[10pt,a4paper]{article}

\usepackage{fullpage}

\title{Proposal for advanced content discovery by a combination of linked data
and data mining techniques}

\author{Ross Fenning}

\date{}

\begin{document}

\maketitle \thispagestyle{empty}

\section{Introduction}

Media companies produce ever larger numbers of articles, videos, podcasts,
games, etc. -- to which it is commonly referred as ``content''. A successful
content-producing website not only has to develop systems to aid producing and
publishing that content, but there are also calls to engineer effective
mechanisms to aid consumers in finding that content.

Approaches used in industry include providing a text-based search, hierarchical
categorisation (and thus navigation thereof) and even more tailored recommended
content based on past behaviour or content enjoyed by friends (or sometimes
simply other consumers who share your preferences).

\section{Problems}

There are several technical and conceptual problems with building effective
content discovery mechanisms, including:

\begin{itemize}

\item Large organisations can have content across multiple content management
systems, in differing formats and data models. Organisations face a large-scale
enterprise integration problem simply trying to gain a holistic view of all
their content.

\item Many content items are in fairly opaque formats, e.g video content may be
stored as audio-visual binary data with minimal metadata to display on a
containing web page. Video content producers may not be motivated to provide
data attributes that might ultimately be most useful in determining if a user
will enjoy the video.

\item Even with more accessible formats such as text content we need information
retrieval and natural language techniques to provide searchable data structures.
There is not a clear leap from natural language content to more sophisticated
formats suitable for more advanced data mining or machine learning approaches.

\item Content is being published continuously, which means any search or
discovery system needs to keep up with content as it is published and process it
into the appropriate data structures. Any machine learning previously performed
on the data set may need to be re-run.

\end{itemize}

\section{Proposal}

In this proposal, I offer the following hypotheses for research:

\begin{itemize}

\item Research and software tools around the concept \emph{Linked Data} can aid
us in rapidly acquiring a broad view (perhaps at the expense of depth) of an
organisation's content whilst also providing a platform for simple enrichment of
that content's metadata.

\item We can establish at least a naïve mapping of an RDF graph of content items
and other entities of interested. With such a mapping, we can explore applying
machine learning -- particularly unsupervised learning -- across an
organisation's whole content corpus in a relatively short time frame.

\item This rapid ease of both acquiring data and providing initial insights with
established algorithms is particularly in line with the principles of the
\emph{Agile} and \emph{Lean} methodologies and we can thus hope to engineer a
prototype-quality web page visualisation of the outcomes from initial data
mining attempts. Such a prototype is arguably a suitable first iteration for a
new content discovery portal, application or website feature an organisation
might wish to develop and release to its visitors.

\item We can adapt established machine learning approaches such as clustering
for data published continuously in real time.

\end{itemize}

\section{Outcomes}

The proposed outcomes are outlined here in stages to allow the possibility of
only a subset thereof being achieved in the given time.

\end{document}

\begin{comment}

This proposal essentially brings together the fields of linked data/semantic web
and data mining, with some possibilities of stepping into natural language
processing. The premise is to handle finding and curating amongst large
quantities of online web content produced by media entities like the BBC. We can
use linked data and the semantic web to acquire representations of those content
items at scale and enrich what we know about them by following links, unifying
with other data sources and by performing logical inference (perhaps even using
linguistic semantics to aid what we can infer).

Once we have adequately acquired data on content, we can look at machine
learning techniques and visualising consequent models to users as a discovery
portal of sorts. My first instinct is to look at clustering content items so as
to provide onward journeys to similar items that might not otherwise have been
seen by users (and perhaps conversely make use of dissimilarity to provide more
tangential journeys). This is meant as a divergence from -- but a useful adjunct
to -- information retrieval approaches where similarity focuses on keywords
extracted from text. I may find time to explore the benefits of other machine
learning techniques.

I'm seeing some novelty in 1) handling unsupervised learning and enrichment of
data as it arrives in real time (content is consistently published as opposed to
working on static data sets); and 2) the benefits of the linked data research
and tools that already exist that can provide methods for rapid acquisition and
transformation of data ahead of machine learning. There are also possibilities
to touch on natural language research and entity extraction to improve the
enrichment side of the system to improve the machine learning side further.

\end{comment}
