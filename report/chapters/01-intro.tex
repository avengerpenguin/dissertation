\chapter{Introduction}

Media companies produce ever larger numbers of articles, videos, podcasts,
games, etc. -- commonly collectively known as ``content''. A successful
content-producing website not only has to develop systems to aid producing and
publishing that content, but there are also demands to engineer effective
mechanisms to aid consumers in finding that content.

Approaches used in industry include providing a text-based search, hierarchical
categorisation (and thus navigation thereof) and even more tailored recommended
content based on past behaviour or content enjoyed by friends (or sometimes
simply other consumers who share your preferences).

\section{Problems}

There are several technical and conceptual problems with building effective
content discovery mechanisms, including:

\begin{itemize}

\item Large organisations can have content across multiple content management
systems, in differing formats and data models. Organisations face a large-scale
enterprise integration problem simply trying to gain a holistic view of all
their content.

\item Many content items are in fairly opaque formats, e.g. video content may be
stored as audio-visual binary data with minimal metadata to display on a
containing web page. Video content producers may not be motivated to provide
data attributes that might ultimately be most useful in determining if a user
will enjoy the video.

\item Content is being published continuously, which means any search or
discovery system needs to keep up with content as it is published and process it
into the appropriate data structures. Any machine learning previously performed
on the data set may need to be re-run.

\end{itemize}

\section{Hypothesis}

The following hypotheses are proposed for gaining new insights about an
organisation's diverse corpus of content:

\begin{itemize}

\item Research and software tools around the concept of \emph{Linked Data} can
aid us in rapidly acquiring a broad view (perhaps at the expense of depth) of an
organisation's content whilst also providing a platform for simple enrichment of
that content's metadata.

\item We can establish at least a na\"ive mapping of an RDF graph representing a
content item to an attribute set suitable for data mining. With such a mapping,
we can explore applying machine learning -- particularly unsupervised learning
-- across an organisation's whole content corpus.

\item Linked Data and Semantic Web \emph{ontologies} and models available can
provide data enrichment beyond attributes and keywords explicitly avaiable
within content data or metadata.

\item We can adapt established machine learning approaches such as clustering
for data published continuously in real time.

\item Many content-producers currently enrich their web pages with small
amounts of semantic metadata to provide better presentation of that content
as it is shared on social media. This enables simple collection of a full
breadth of content with significantly less effort than direct integration
with content management systems.

\end{itemize}
