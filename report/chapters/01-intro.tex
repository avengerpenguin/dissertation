\chapter{Introduction}
\label{chp:intro}

Media companies produce ever larger numbers of articles, videos,
podcasts, games, etc. -- commonly collectively known as ``content'' --
on the World Wide Web. A successful content-producing website not only
has to develop systems to aid producing and publishing that content,
but there are also demands to engineer effective mechanisms to aid
consumers in finding that content.

Approaches used in industry include providing a text-based search,
hierarchical categorisation (and thus navigation thereof) and even
more tailored recommended content based on past behaviour or content
enjoyed by friends (or sometimes simply other consumers who share your
preferences).

\section{Problems}
\label{sec:intro-problems}

There are several technical and conceptual problems with building
effective content discovery mechanisms, including:

\begin{itemize}

\item Large organisations can have content across multiple content
  management systems, in differing formats and data models.
  Organisations face large-scale enterprise integration problems
  simply trying to gain a holistic view of all their content.

\item Many content items are in fairly opaque formats, e.g. video
  content may be stored as audio-visual binary data with minimal
  metadata to display on a containing web page. Video content
  producers may not be motivated to provide data attributes that might
  ultimately be most useful in determining if a user will enjoy the
  video.

\item Content is being published continuously, which means any search
  or discovery system needs to keep up with content as it is published
  and process it into the appropriate data structures. Any machine
  learning previously performed on the data set may need to be re-run.

\end{itemize}

Many of these problems are felt within the BBC and are potentially
issues in many similar organisations. The BBC website officially
launched in 1997 and many articles from that year are still available
in archived form. Over the last 20 years, multiple content management
systems were created and retired such that today even the active
website is served by a mix of emerging, cutting-edge platforms and
systems that have been running for a number of years.

Technologies such as the World Wide Web and the maturity of the
media industry in its adoption in the last decade creates both an
opportunity to give consumers access to a substantial amount of
media content and a challenge in trying to make cohesive sense of
a diverse range of content, perhaps produced in an organisation that
was traditionally very fragmented as the BBC was.

\section{Hypothesis}

The following hypotheses are proposed for gaining new insights about
an organisation's diverse corpus of content:

\begin{itemize}

\item Research and software tools around the concept of
  \emph{Linked Data} can aid us in rapidly acquiring a broad view
  (perhaps initially at the expense of depth) of a organisation's
  content while also providing a platform for simple enrichment of
  that content's metadata.

\item We can establish at least a na\"ive mapping of an RDF graph
  representing a content item to an attribute set suitable for data
  mining. With such a mapping, we can explore applying machine
  learning -- particularly unsupervised learning -- across an
  organisation's whole content corpus.

\item Linked Data and Semantic Web \emph{ontologies} and models
  available can provide data enrichment beyond attributes and keywords
  explicitly available within content data or metadata.

\item Many content-producers currently enrich their web pages with
  small amounts of semantic metadata to provide better presentation of
  that content as it is shared on social media. This enables simple
  collection of a full breadth of content with significantly less
  effort than direct integration with individual content management
  systems.

\end{itemize}

It is proposed also that all the data acquisition, enrichment and
mappings can be implemented in a general way when semantics and
linked data are used. The standardisations offered by the Semantic
Web and Linked Data research fields provides uniform interfaces and
patterns of data manipulation such that any software produced can be
applied to all of an organisation's content without consideration to
any particular class or type of item. This also does not preclude
including custom business rules later either, where domain experts
can give more guidance to treat data more intelligently.

The core principle in this research is that a generally-applicable
software system provides the best trade-off in terms of processing
the full breadth of content with the least amount of engineering
effort. This can allow media organisations to benefit from things
such as insights from machine learning across their whole corpus with
near-immediate results rather than embarking on large scale enterprise
integration projects over several months or years.
