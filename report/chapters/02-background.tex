\chapter{Background}

This chapter discusses some of the existing research and technologies around
machine learning, RDF and combining them. It also covers some of the advantages
of using linked data and RDF in an enterprise setting and what tools and
approaches are well-defined enough that a corporation could build on top of
them rapidly.

Data mining activities such as machine learning rely on structuring data as
\emph{feature sets}\cite{bishop2006pattern} -- a set or vector of properties or
attributes that describe a single entity.
The process of \emph{feature extraction}
generates such feature sets from raw data and is a necessary early phase for
many machine learning activities.

The rest of this chapter will show:

\begin{enumerate}
\item that extracting feature sets from RDF\footnote{http://www.w3.org/TR/PR-rdf-syntax/} graphs can be done elegantly and follows naturally from some
previous work in this area; and
\item that the RDF graph is a suitable and even desirable data model for content
metadata in terms of acquiring, enriching and even transforming that data ahead
of feature extraction.
\end{enumerate}

\section{Data Mining}

TODO

\section{RDF and Feature Extraction}
\label{sec:rdf-and-features}

The RDF graph is a powerful model
for metadata based on representing knowledge as a set of
subject-predicate-object \emph{triples}. The query language, SPARQL, gives us a
way to query the RDF graph structure using a declarative pattern and return a
set of all variable bindings that satisfy that pattern.

For example, the SPARQL query in Listings~\ref{lst:sparqlfoaf}
queries an RDF graph that contains contact information and returns the
names and email address of all ``Person'' entities therein.

Notably, Kiefer, Bernstein and Locher\cite{kiefer2008adding} proposed a novel
approach called SPARQL-ML -- an extension to the
SPARQL\cite{segaran2009programming} query language with new keywords to
facilitate both generating and applying models. This means that the system
capable of parsing and running standard queries must also run machine learning
algorithms.

Their work involved developing an extension to the SPARQL query
engine for \emph{Apache Jena}\footnote{https://jena.apache.org/} that integrates
with systems such as \emph{Weka}\footnote{http://www.cs.waikato.ac.nz/ml/weka/}.
A more suitable software application for enterprise use might focus solely on
converting RDF graphs into a neutral data structure that can plug into arbitrary
data mining algorithms.

\begin{lstlisting}[label=lst:sparqlfoaf,caption={Example SPARQL query for people's names and email addresses},language=sparql]
PREFIX foaf: <http://xmlns.com/foaf/0.1/>
SELECT ?name ?email
WHERE {
  ?person a foaf:Person.
  ?person foaf:name ?name.
  ?person foaf:mbox ?email.
}
\end{lstlisting}

If we consider an RDF graph, $g$, to be expressed as a set of triples:

\begin{displaymath}
  (s, p, o) \in g
\end{displaymath}

\noindent this query could then be
expressed as function $f: G \rightarrow (S \times S)$ where $G$ is the set of
all possible RDF graphs and $S$ is a set of all possible strings.
This allows the result of the
SPARQL query to be expressed as a set of all SELECT variable bindings that
satisfy the WHERE clause:

$$
q(g,n,e) = \exists p . (p, type, Person) \in g\ \land (p, name, n) \in g \land (p, mbox, e) \in g
$$

$$
g \in G \models f(g) = \{(n, e) \subseteq (S \times S) \: | \: q(g,n,e)\}
$$

This could be generalised to express a given feature set as
vector $(a_1, a_2, ..., a_n)$:

$$
g \in G \models (a_1, a_2, ..., a_n) \in f(g)
$$

\noindent and in the case where all $a_k \in f(g)$ are literal (e.g. string or
numeric) values, we can thus consider a given SPARQL query to be specific
function capable of feature extraction from any RDF graph into sets of
categorical or numeric features.

\begin{lstlisting}[label=lst:sparqlabout,caption={SPARQL query to determine what },language=sparql]
PREFIX rdf: <http://www.w3.org/1999/02/22-rdf-syntax-ns#>
SELECT ?topic
WHERE {
  ?article rdf:about ?topic .
}
\end{lstlisting}

This might allow a query that extracts a country's population, GDP, etc.
provide feature extraction for learning patterns in economics, for example.
However, this is limited to features derived from single-valued predicates
with literal-valued ranges. It is not clear how to formulate a query that
expresses whether or not a content item is about a given topic.

In the RDF
model, it would be more appropriate to use a query like that in
Listings~\ref{lst:sparqlabout} where for a given $?article$ identified by
URI, we can get a list of URIs identifying concepts which the article mentions.
Such a query might be expressed as function $f': G \rightarrow \mathcal P(U)$ where $U$ is
set of all URIs such that:

$$
g \in G \models f'(g, uri) = \{t \: | \: (uri, about, t) \in g\}
$$

An approach of generating attributes for a given resource was proposed by
Paulheim and F\"urnkranz\cite{paulheim2012unsupervised}. They defined specific
SPARQL queries and provided case study evidence for the effectiveness of
each strategy.

Their work focused on starting with relational-style data (e.g. from a
relational database) and using \emph{Linked Open Data} to identify entities
within literal values in those relations and generated attributes from
SPARQL queries over those entities.

For a large content-producer, there is a more general problem where many content
items do not have a relational representation and the content source is a body
of text or even a raw HTML page. However, the feature generation from Paulheim
and F\:urnkranz proves to be a promising strategy given we can acquire an RDF
graph model for content items in the first place.

\section{RDF in the enterprise}
\label{sec:linked-enterprise-data}

TODO
