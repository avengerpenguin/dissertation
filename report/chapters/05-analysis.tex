\chapter{Results and Analysis}
\label{chp:analysis}

In chapter~\ref{chp:design}, a data pipeline is derived that -- at
the logical maximum (see figure~\ref{fig:maximal-pipeline}) could
generate feature sets using any of 112 combinations of extraction
and enrichment techniques. This is based on the product of using or
not using any of three described extraction approaches (less the case
where no extraction is performed) and enabled or disabling any of
four distinct enrichment approaches (retaining the case where no
enrichment is performed).

More practically, in chapter~\ref{chp:implementation}, the
implementation of a real system for evaluating a subset of only
fourteen combinations of approaches is described. Therefore there
are fourteen resultant sets of clusters of the same BBC web content
produced for comparative analysis.

In the following sections of this chapter, an objective and technical
analysis of the clusters is performed, leading to a critical review
of each individual extraction and enrichment approach, based on how
each respective technique appears to contribute to the results.

In chapter~\ref{chp:evaluation}, there is an evaluation of the
overall system based
on qualitative data gathered by a survey of real human users. That
evaluation is more focused on the suitability of the holistic use of
semantics and machine learning in the real world application of
suggesting related media content to web users.

Conversely, in this chapter, we take a much narrower focus on the
shape, size and quality of clusters themselves.

\section{Comparison of Clusters Produced}
\label{sec:anal-comparison}

In this section, we take a walkthrough of some of the clusters
produced based on what seems interesting. Even reviewing a handful
of the results shows some insights into how the different semantic
extraction approaches have behaved and we can see some clear
evidence where the process is generally working and where it is
making mistakes.

It is not feasible to analysis every result here in this way, but
chapter~\ref{chp:evaluation} describes how every approach was
evaluating against every other approach The sections following
this one in this chapter try to give a summary of advantages and
disadvantages we can see in each technique ahead of the
qualitative evaluation.

The most stark difference between the results produced is the large
range in terms of number of clusters produced for each approach
respectively.

\begin{table}[h]
  \centering
  \caption{Number of clusters produced by each combination of approaches}
  \label{tbl:cluster-counts}
  \begin{tabular}{lll}
    & Not enriched & Enriched \\
    Deference                                     & 5            & 249      \\
    Entity Extraction                             & 35           & 32       \\
    Hyperlink Relationships                       & 2            & 56       \\
    Dereference and Entity Extraction             & 329          & 340      \\
    Entity Extraction and Hyperlink Relationships & 199          & 233      \\
    Dereference and Hyperlink Relationships       & 53           & 117      \\
    All three extractions                         & 184          & 223     
  \end{tabular}
\end{table}

Table~\ref{tbl:cluster-counts} shows the number of clusters produced
by each combination of techniques. The range here is explained by
some variants of the system produced few, very large clusters and
some a large number of smaller clusters.

\subsection{Few, Large Clusters}
\label{sec:few-large}

Which of the two extreme outcomes is more suitable is arguably a
matter for real evaluation, but it is perhaps reasonable to claim that
clusters which are too large are less useful for choosing related
content items to suggest to users. In the case of the unenriched graphs
obtained via hyperlink relationships only, one of the two clusters
produced contains 2,323 items -- a number that risks
``related content'' suggestions being too erratic and random.

A deeper inspection of that largest cluster reveals every item was
deemed to be related to an FAQ page on how to share BBC pages on
social media websites -- a page that appears to be linked to from
every BBC programme information page or iPlayer catch-up video page.
This is clearly not a good indication of the content of the page
and raises two immediate issues:

\begin{enumerate}
\item the system would benefit from further heuristics that avoid
  inferring too many relationships from ancillary pages such as FAQs
  that are unrelated to the content itself; and
\item the nature of agglomerative, hierarchical clustering means
  that this system variant spent a large amount of time grouping all
  pages that link to this FAQ page at the expense of other,
  potentially useful clusters. This is a consequence of hierarchical
  clustering being a greedy algorithm.
\end{enumerate}

Further discussion of the merits and problems with hyperlink
relationships is covered in section~\ref{sec:anal-hyperlink}.

\subsection{More, Small Clusters}

If we look at the other end of the spectrum, the system variant
that produced the most clusters was the enriched union of dereference
and entity extraction. With so many clusters, the distribution of
sizes is better shown by the histogram in figure~\ref{fig:sizes-3-1}.
Here we can see the results are dominated by clusters with 25 or
fewer items.

\begin{sidewaysfigure}
  \begin{tikzpicture}[xscale=3,yscale=2]
    \begin{axis}[
        ybar,
        ymin=0
      ]
      \addplot +[
        hist={
          bins=12,
          data min=0,
          data max=300
        }   
      ] table [y index=0] {../data/sizes-3-1.csv};
    \end{axis}
  \end{tikzpicture}
  \caption{Distribution of cluster sizes produced by the enriched union of dereference and entity extraction}
  \label{fig:sizes-3-1}
\end{sidewaysfigure}

An exploratory insight into the makeup of these clusters is shown
in table~\ref{tbl:top-features-3-1} in which the ten most common
features are shown for the one of the smallest, one of the biggest
and one of the mean-sized clusters.

\begin{sidewaystable}[p]
  \centering
  \begin{tabular}{|l|l|l|}
    $N=2$                                 & $N=9$                                                   & $N=265$                                     \\
    \hline
    ogp:image="[\ldots]"                  & rdf:type\_schema:RadioEpisode=true                      & md:item\_rdf:nil=true                       \\
    foaf:topic\_dbpedia:BBC\_iPlayer=true & schema:url="[\ldots]"                                   & foaf:topic\_dbpedia:Digg                    \\
    rdf:type\_schema:RadioEpisode=true    & foaf:topic\_dbpedia:JavaScript                          & twitter:card:"summary\_large\_image"        \\
    rdfa:usesVocabulary\_schema=true      & meta:twitter:card="summary"                             & foaf:topic\_dbpedia:LinkedIn                \\
    meta:twitter:card="summary"           & foaf:topic\_dbpedia:Listen\_(Beyonc\'e\_Knowles\_song)" & meta:apple-mobile-web-app-title="BBC Sport" \\
  \end{tabular}
  \caption{Ten most common features across the smallest, the largest and a mean-sized cluster produced by the enriched union of dereference and entity extraction}
  \label{tbl:top-features-3-1}
\end{sidewaystable}

\subsubsection{Smallest Cluster}
\label{sec:smallest-cluster}

The smallest cluster with 2 items in table~\ref{tbl:top-features-3-1}
seems to contain two pages that mention BBC iPlayer (the
\texttt{foaf:topic} predicate suggests this was found via entity
extraction) and whose primary content is an item with \texttt{rdf:type}
``RadioEpisode'' in the Schema.org\footnote{http://schema.org/}
vocabulary. This shows hints of a strong complement between semantic
web data and entity extraction confirming that textual content on the
page correlates with any semantics also embedded in the page.

Interestingly, this cluster is also strengthened by the fact that
both pages have a common thumbnail (the \texttt{ogp:image} predicate
indicates a thumbnail when used to share on Facebook). An examination
of the two content pages in question confirms that they are two
episodes of the same ``5 Live Breakfast'' programme on the
\emph{BBC Radio 5 live} radio channel.

The other two predicates for the small cluster are less clear without
some explanation as to their origin. The \texttt{meta:twitter:card}
predicate is a
documented\footnote{https://dev.twitter.com/cards/types/summary}
HTML meta tag that to indicate a page has included further meta tags
to control which metadata to include in the so-called
\emph{Summary Cards} Twitter generates as a summary of page a user
is sharing in a post to the website. This overlaps some of the
features provides by the \emph{Open Graph protocol}\footnote{http://ogp.me/}
vocabulary used by Facebook for similar means.

This overlap between Facebook and Twitter highlights both:

\begin{itemize}
  \item a failure
    for industry to adopt (or for academia to convince them to adopt)
    semantic web initiatives such as RDFa,
    Microdata, Microformats or Schema.org -- designed to provide
    machine-readable semantics in ways richer than simple meta tags; but
    also
  \item a critical opportunity for research and development in semantics
    to build on top of the real, commercial pressure created by Facebook,
    Twitter, Google et al to enrich pages with semantic data so that
    content providers can control how their content is summarised when
    shared on their platforms and also how their content is analysed for
    searching and extrapolation of real time trends in social media
    conversation.
\end{itemize}

That predicates for Facebook and Twitter are appearing in this
experiment so easily also shows evidence that the BBC is among the
media organisations that now have content producers providing
semantic enrichment alongside their content production or at least
that automated generation of in-page semantics from metadata already
present in content management systems is being prioritised alongside
other work to develop features more visible to human users. It is not
unreasonable to suppose this drive is at least partly driven by
content producers' and content promoters' desire to control how
content is displayed and discovered on third party social media
platforms.

\subsubsection{Mean-sized Cluster}

Returning to the mean-sized cluster in table~\ref{tbl:top-features-3-1},
we see a cluster where the most common items are identified as pages
each for a single episode. There is no indication in the top five
features that they are episodes of the \emph{same} programme, however.
This could well be a strength of this system variant in that it is
able to suggest other radio programmes on the basis it has understood
they all share a common type, but it is open to suggesting episodes
of other programmes for greater diversity. In fact, this may be a
necessary feature in any ``related content'' feature as it could
be safely assumed that any BBC page for a single episode of a radio
programme already contains clear and sufficient navigation links to
find more episodes of the same programme; machine learning adds no
value to the user here.

Again, there is potential to call out a real benefit in the
combination of in-page semantics for this ``hard'' categorical
data (i.e. the page's content is of type ``Radio Episode'') with
entity extraction for a looser, schemaless vocabulary of words that
more qualitatively describe the topics covered in the content.

In this case, the topics extracted appear to be the JavaScript
programming language and the song \emph{Listen} by music artist
Beyonc\'e Knowles. A manual review of the pages involved finds them
to be radio episode pages again, but with no clear indication that
they mention either topic. Deeper analysis of the HTML source for
these pages finds two ``mistakes'' the system has made:

\begin{itemize}
\item the word JavaScript appears in the page only as part of the
  MIME type \texttt{text/javascript} in markup around JavaScript
  \emph{code} supporting the page; and
\item the word Listen appears on \emph{every} BBC page concerning
  a radio programme due to the fact they all contain a ``Listen''
  button allowing the user to listen to the live radio stream for
  that particular radio station. DBpedia Spotlight has assumed this
  word is a mention of a song titled \emph{Listen}.
\end{itemize}

\subsubsection{Largest Cluster}

Finally, the largest cluster has unique quirks of its own. The most
popular feature across all members was a rather strange
\texttt{md:item\_rdf:nil=true} whose original is unclear with some
explanation. This triple appears to have originated in the RDFLib
Python library as an artefact of the Microdata parser (run on any
HTML response) having found no Microdata items in the page. This is
likely to be a bug in the library where it may have been more useful
not to say anything.

The result of this bug or unexpected behaviour is that variants
of the data pipeline that use deferencing as an extraction method will
have this propery for any page that does not use Microdata simply
because of the choice of RDF parsing library. There appears to be
enough mix of BBC content pages that do and do not use Microdata such
that this feature passed the feature selection filter that should
have removed any feature with the same value across the whole corpus.

This is part of a bigger behavioural pattern where embedded semantics
appear to tell us a lot about how pages are built as well as
describing the content contained therein. This is not necessarily a
drawback and is discussed further in section~\ref{sec:anal-deref}.

Two other features in the five most common across the cluster suggest
many of the content items mention the social network, LinkedIn, and
news aggregator, Digg. These come from a similar issue with the
\emph{JavaScript} and \emph{Listen} entities for the mean-sized
cluster in that they are mentioned in the page as part of controls
inviting users to ``share'' the content on social media sites. These
are also clear false positives that reinforce the importance of
ensuring entity extraction is performed in relevant textual
information. This is discussed in section~\ref{sec:anal-entity}.

The fifth most common feature for this cluster is genuinely something
informative about the content in the page -- it appears to be a meta
tag informing of a title to use for the page on iOS devices. This is
again likely to be an artefact of how the BBC Sport section of the
BBC website is developed with requirements relating to iOS devices,
but it also -- perhaps inadvertantly -- tells us something semantic
about the page in that it is part of BBC Sport.

This may not be groundbreaking insight as there are clearly more
definitive ways to distinguish BBC Sport content (e.g. all the URLs
paths start with the word ``sport''), but it starts to showcase a
potential advantage of unsupervised learning on semantics that it
can use subtle clues based on how the pages are constructed to
determine something about them -- if this is deemed useful, that is.

It is also a showcase of how an unsupervised data mining process can
be said to have ``learned'' that these are all BBC Sport articles
without having to be instructed through bespoke business rules. This
hints at some need for further research as to whether enterprises
can truly adopt semantics and data mining to derice business rules
that survive organisational and systems changes due to being less
brittle than bespoke enterprise integration software based on static
business rules (i.e. rules that are true today but need updating when
the business context changes or risk breaking).

The more qualitative evaluation in chapter~\ref{chp:evaluation} will
provide some idea of which properties ultimately end up actually being
useful.

\subsection{Impact of Noise}
\label{sec:impact-of-noise}

The previous section highlighted many cases of ``false positives''
and errors made by this experimental system. The intent of the
designed system is to provide an
initial baseline capable of processing any piece of media content,
perhaps at the expense of the depth that would be achieved by explicit
data adapters developed against an internal data API. Thus it is
useful to analyse where the system performs less well because a goal
for any initial, baseline application should be to evolve and grow it
to adapt to errors as they find them.

An open question still for further research is whether these errors
are real barriers to gaining useful insight via data mining. The
open world assumption and loose nature of the RDF model lends itself
well to embracing the patchy and noisy nature of the open web of data
and that can be a strength even across a single organisation's
own content -- particularly in large and fragmented organisations.

In a rigid, closed-world approach, we may see bespoke data adapters
fail altogether the first time a team responsible for a data API
changes, say, the XML structure of its query responses without
notifying all clients. We may also see data mining processes lacking
the ability to find trends in the entire corpus since the data
ingest written specifically for custom APIs has only been built for
a subset of all the data stores in the organisation.

The nature of machine learning algorithms is they are intended to
learn for themselves which data properties are useful and which are
not. In a classification model, we would expect the learning to
place more weight on features that produce the answers desirable
by those that use the model and some of the noise naturally falls
away.

In the current experiment around unsupervised clustering, not only
was feature selection implemented to throw away features that do not
vary and features that vary too much, but clustering algorithms
should favour where items where they share a number of properties, not
group content together simply because it shares one erroneous
attribute.

Thus it can be concluded that the mere existence of noise is not
enough to fault a particular configuration of our semantic data mining
system, but there is a need to answer the question as to whether the
noise is enough to overwhelm and confuse any machine learning
attempts. The question we look to answer in this research is which
of the variants of this experiemental system are more or less
susceptible to noise and errors and recommend how we might research
overcoming any false positives that do arise.

At the very least, there are variants of the system that are clearly
failing very quickly, e.g. the hyperlink relationships producing only
two, unusably large clusters shows a clear need to overcome the causes
of those problems before even considering using it in a real system.

\section{Strengths and Weaknesses of Different Approaches}

The analysis in the previous section walks through some exploratory
reviews of a small number of clusters, whereas in this section an
overview of how each approach appears to show strength and failures
is given.

\subsection{Embedded Semantics Extraction}
\label{sec:anal-deref}

Some of the analysis in section~\ref{sec:anal-comparison} touched on
some of the side effects of semantics being tied to \emph{how} a page
was built rather than \emph{what} the page is about. It was also
shown semantics are being included due to  pressures from social media
organisations for content
providers to describe summaries of their content for better display
and discovery. Based on this and other observations in the clusters
produced, clustering based on deferencing URLs and reading their
embedded semantics has the following advantages:

\begin{itemize}
\item Many media companies are providing semantic data for the
  purposes of display and other features in a social media context.
  This means there is a base level of metadata to be extracted
  without asking those building the pages to add in more semantic
  markup.
\item Some parts of the BBC website have chosen to include
  much richer semantics. This is observed in
  section~\ref{sec:anal-comparison} where pages about radio
  episodes where described as such. Some parts of the BBC Sport
  website also mark up entities such as sport teams as well.
\end{itemize}

We can also observe the following challenges or drawbacks:

\begin{itemize}
\item There can be a lot of noise from meta information about the
  semantics, e.g. the RDFLib library declaring that a page contains
  \emph{no} Microdata as observed in section~\ref{sec:anal-comparison}
  or RDFa declarations that inform that the page is using a particular
  vocabulary by default
  \footnote{The feature
    \texttt{rdfa:usesVocabulary\_http://schema.org/=true} is very
  common}
\item Some parts of the BBC website provide only basic metadata
  (e.g. for social media as above) and some have chosen to include
  far richer semantics. This leads to items varying substantially in
  terms of the size of the RDF graphs we can extract. The
  unbalance show signs of causing a bias where clusters may not be
  likely to a mixture of such items, instead preferring to group
  the semantically-rich items together for example.
\item There are very general-purpose assertions such as
  \texttt{ogp:site\_name="BBC"} provided simply to tell Facebook that
  a given page is part of the overall BBC website. This is not useful
  if the whole corpus is from the BBC website, but does not get
  filtered out due to pages being built differently not all
  development teams behind individual parts of the website choosing
  to include that property. Thus the varied usage and non-usage of
  this feature implies to a machine learning algorithm there is
  information in the feature's presence or non-presence.
\end{itemize}

The last drawback above suggests
a bigger question here as to whether there is a more general
challenge with machine learning and the open world assumption of
linked data and semantics. A classification model surely has to
choose between non-membership or membership of a category with
some degree of certainty, so will mistake the lack of knowledge
of an attribute that would place it in a category as if that item
genuinely lacks that defining feature.

Similarly, a clustering
algorithm can only group items on what is known. Perhaps if features
are lossy in a random sense, then other features will still link
two related items, but if, say, a whole section of a website fails
to provide semantics on genre classification of its content, then
content items in that section will never be paired up with pages
in another section of the site that does provide useful information
above genres.

It should be noted that this is still a problem in trying to integrate
information between two bespoke databases that follow a closed-world
assumption. If one content management system encourages authors to
categories their work into genres and another does not, then we
still cannot link up content. This situation is arguably worse in that
two systems that \emph{do} provide genre information, but using wholly
different names for the properties and a different taxonomy of values,
still cannot easily unify the data models without explicit and
possibly extensive rules translating between the two.

It could be said that the Linked Data concept and by extension the
idea of Linked Enterprise Data allows two different data sets to be
developed independently for their own use cases so long as they
(or a third party) is encourage to provide additional semantics that
link the vocabularies. This is even easier still if the two systems
use different, but well-known vocabularies, as the mappings may
already exist.

\subsection{Entity Extraction}
\label{sec:anal-entity}

Inferring semantics from entity extraction has some clear advantages:

\begin{itemize}
\item Extracting real things from textual content gives a far
  stronger feel for the topics covered therein. Note this is
  essentially the basis of any keyword-driven search engine such
  as Google.
\item Implementation in the enterprise is relatively simple given
  the vast number of technologies already available and the long
  history of research into entity extraction. Reuse of an existing
  solution -- as implemented in this research -- is sufficient for
  many use cases.
\end{itemize}

However, there are still many drawbacks and challenges to overcome:

\begin{itemize}
\item Entity extraction classically suffers from a disambiguation
  challenge (e.g. is it Paris, France or Paris, Texas?), but there
  is also much literature on solving and overcoming these problems.
  The na\"ive, simple usage of DBPedia Spotlight used in this
  experiment benefitted from no customisation of the parameters used
  nor from any attempts to solve ambiguity problems.
\item It is less clear as to whether we can narrow down which
  are the primary topics or entities in a piece of content. Some
  search engines use heuristics such as preferring entities in
  titles and opening paragraphs, but this is still a fuzzy approach.
  In this research, a simple ``topic'' predicate from the FOAF
  vocabulary was used and thus gave equal status to all entities,
  assuming they are each equally a topic of the content. Better
  research in this area might provide more effective ways to discern
  a ``topic'' relationship from, say, a looser ``mentions''
  relationship.
\item All entity extraction technologies mentioned so far have
  focused around text content only. The ability to extract topics
  from audio or video content is much more complex and is the subject
  of whole other fields of research. Some very promising techniques
  were prototyped by Raimond et al.\cite{raimond2014bbc} to extract
  semantic data from the BBC World Service archive using a novel
  combination of text-to-speech analysis, machine tagging and
  crowdsourced user feedback at scale. There could be much
  opportunity to applying such research more generally, for example
  across BBC iPlayer catch-up content, which can then feed into
  better data mining.
\item Effective extraction of the content ``body'' text must be
  performed to prevent entities being found in supporting page
  furniture such as site-wide navigation links. A single failure to
  remove an element common to multiple pages will result in all
  those pages being ``tagged'' with that concept. 
\end{itemize}

The final drawback is worsened
when elements are across just one class of pages (as observed with
the \emph{Listen} problem in section~\ref{sec:anal-comparison}) as
at least entirely global entities will not pass feature selection.
When a navigation element only appears in part of the corpus, the
system might erroneously decide this is an informative feature.

As noted in section~\ref{sec:anal-deref} with semantics that expose
the nature of the construction of the page, a supporting element on
a particular class of pages might actually be useful information. To
refer back to the \emph{Listen} example -- where a ``Listen'' button
for radio stations on radio episode catch-up pages is mistaken for
a song with that title -- it can be argued that the clustering model
``learned'' something common between all these pages. It could be
said that we need
not worry about the technically incorrect semantics that it ``thinks''
the pages are about that song if the end result is a useful feature
that understands there is some commonality between radio episode
catch-up pages and it is suitable to consider them as related content
suggestions when a user starts on such a page.

This is an overall advantage to applying semantics to data mining:
we do not need our machines to gain a perfect ``understanding'' of
the content so long as the so-called understanding is
\emph{good enough} that applications of the models -- such as
suggesting related content -- ultimately appear to be behaving
intelligently in the average case.

This conclusion is reminiscent of John Searle's famous
\emph{Chinese Room} thought experiment\cite{searle1980minds}
that illustrates a computer
programme's potential to appear intelligent even when following
deterministic rules and lacking the deep understanding and
consciousness that we perceive as uniquely human.

\subsection{Hyperlink Relationships}
\label{sec:anal-hyperlink}

Exploratory analysis does not highlight any major strengths from
hyperlink exraction alone. Common properties centre almost entirely
around links to common pages such as FAQs, help pages, etc. due to the
lack of constraints in the system implemented to narrow the extraction
to ``interesting'' links from within the content itself.

This extraction technique might well perform better on a website
such as Wikipedia where articles contain a large number of links to
other articles, but BBC content employs links within article bodies
very rarely. Other news websites such as The Guardian make more
use of hyperlinks within articles to other articles and topic
aggregate pages, so results for content from The Guardian website
might look very different to those for BBC content.

\subsection{Enrichment by Dereference}
\label{sec:anal-enrichment}

The most striking observation of all clusters from enriched RDF
graphs is that they can have orders of magnitude more features than
their unenriched counterparts. This was especially true in the case
of enriched RDF graphs obtained via entity extraction. This is due
to the fact that any lengthy article may yield ten to fifty entities,
perhaps more. As DBPedia URLs were used for entity tags in this
experiment and DBPedia can have very rich graphs for popular entities,
it is not difficult to extrapolate an assumption that the final
feature sets will be very large.

The size of these feature sets makes it challenging to analyse in an
exploratory sense, but the comparative evaluation in
chapter~\ref{chp:evaluation} hopefully proves whether the size of the
RDF graphs and resulting feature sets is ultimately harmful to the
data mining process.

Enrichment via dereference was less dramatic for in-page semantics
themselves extracted via dereference. Whilst not called out
specifically in section~\ref{sec:anal-comparison}, a general trend
was for those semantics exclusively to contain literals as values.

This is a consequence of relying on HTML meta tags and other such
mechanisms for social media sites. Richer markup approaches such
as RDFa and Microdata allow for entities identified by IRIs and URLs
to be the objects of any triples, but since these were employed
rarely in relation to metadata for social media, they appeared
less often in the resulting clusters.

Finally, inferred semantics from hyperlinks would solely contain
other BBC page URLs as objects (because the implementation explicitly
filtered for BBC URLs only) and thus gained all the advantages and
difficulties with extraction via dereference. A casual inspection
of the clusters suggests that enrichment here did not therefore add
as much additional value as was the case with entity extraction.

\section{Recommendations for Production Use}
\label{sec:production-recommendations}

It can be argued that building an experimental system full of flaws
and drawbacks is not just more desirable, but essential for research
and learning.
However, if a similar system were to be built as a production-ready
application delivering real benefits to customers in industry, there
is clear need to do more work to overcome practical limitations.

Thus this section concludes this chapter with some recommendations
based on all learnings so far on how a media organisation might
want to start applying data mining of their content using semantics.

\subsection{Embedded Semantics}

The Semantic Web movement provoked some pragmatic
scepticism\cite{marshall2003semantic} over the years, mainly focused
around the expectation that content authors are unlikely to create
alternative data for machines on top of their primary objective of
creating content to be consumed by people.

The findings and analysis presented so far in this chapter suggest
there are modern incentives around summarisation of content
(e.g. Twitter ``cards'' when content is embedded in a post) and
discovery (both Facebook and Twitter show trends in topics discussed
in news items posted to their networks). Recognition of this shift
in incentive and consequently the presence of more metadata than
before does not appear to be noted so much in the academic literature.

Haas et al.\cite{haas2011enhanced} provided evidence in 2011 that
fewer than four percent of pages retrieved via Yahoo US search
contained RDFa markup. In a media organisastion that has no culture
of technical employees with an interest in the semantic web or
linked data, there may be no pages with semantic metadata purposely
added. However, if we embrace meta tags intended for social media
websites, exploratory analysis of BBC metadata in this research
suggests a ubiquity of such semantics driven entirely without the
academic push behind the Semantic Web and Linked Data.

We can therefore see social media annotations as compromise between
benefits of semantic markup and the pragmatic concerns that authors
will not add hidden metadata without clear, immediate benefits. It is
hoped that the reults of this research show that there are some
baseline benefits to an initial round of data mining on those
semantics alone and gaps and failures can create the incentives for
web page authors and engineers to improve their semantics and see
immediate benefits (e.e. their pages get promoted more readily in
pan-site signposting such as sets of ``related content'' suggestions).

In summary, the recommendation for the media industry here is to lead
with building insights based on what data are already in place and
look to improve over time once the incentives to do so are in place.
Gaps to consider are where semantics are missing altogether and a
large gap observed in BBC content alone is a lack of linked data
between distinct areas of the website.

For example, programme
information and catch-up pages are excellent at describing themselves
as being TV or radio programmes and BBC News has the early stages of
pages that aggregate News over a given topic, but there is no
cross-linked data where concepts are covered both in a news article
and in a programme. This will emerge in organisations where distinct
teams and divisions work on very distinct parts of their output, so
any data mining concerned with the whole corpus needs to highlight
issues from lack of cross-linking as much as possible, particuarly
if that data mining is driving a real website feature that will see
immediate improvement if such cross-linking is implemented.

\subsection{Entity Extraction}

The clearest requirement here is to ensure that entity extraction
is run on only the main textual content of any page. With sufficient
experimentation, an organisation might be able to tolerate the false
positives on a more na\"ive implementation that processes the whole
page. Over time, the system could evolve with bespoke content
adaptors common in enterprise integration so as to extract the pure
textual content directly from the source system and thus bypass
the need to remove noise from around the article.

Also, there will be a need to spend time tuning and configuring
the entity extraction to produce the right quantity and quality of
entities for each content item. There is no clear single approach
from this initial research alone and organisations will need to
experiment on their own content.

\subsection{Hyperlink Relationships}

The only conclusion from this particular experiment is that
hyperlink relationships are a poor indicator of the nature of the
content in the BBC content corpus. An organisation might only want to
make use of these relationships if its content authors habitually
include links within articles to other articles or pages.
