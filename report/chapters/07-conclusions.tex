\chapter{Conclusions and Future Work}

This chapter lays out some conclusions from this research around the potential
for semantics being used more in the enterprise, whether semantics combined
with machine learning or data mining has potential in industry and how
organisations should consider implementing this. Finally, some options and
potential for future research are suggested.

\section{Challenges and Opportunitues for Semantics in the Enterprise}

As discussed in section~\ref{sec:production-recommendations} and also more
generally throughout chapter~\ref{chp:analysis}, there is much scope for
enterprises to benefit from using semantics in public-facing pages that display
their own content. This is particulary necessary in larger organisations where
attemping to gain an overview of all the content metadata in a short timeframe
would be a large enterprise integration effort.

The challenges amount to the issues raised around noise (semantic information
not immediately relevant to the content), incomplete information (due to
the open world assumption) and perhaps even erroneous information (embedded
Semantic Web data is not always visible enough for publishers to spot errors.)

Even with these issues, the experiment performed in this research was still
able to take place across a wide range of BBC content without any bespoke
integration against internal data APIs nor a any significant code created
to deal with a particular data source.

The power to use semantics and other web standards to gain a broad overview
of a large set of content should be very appealing to a media organisation,
even if there are issues with that data, assuming their application is in
a use case where less precision of that information is tolerable.

Even where data needed corrected, the software developed even in this
experiment was able to adapt to broad patterns in the data that were
problematic. there is also much indication that engineers in the enterprise
would be to complement the breadth from extracting semantics with the depth
of data obtained directly from data sources. This is achievable with the
set-theoretic nature of RDF graphs and our ability to perform simple set
union operations to join data sources.

As is evidenced by the pressure from social media and networks such as
Facebook and Twitter, publishers will include better semantics if the right
incentives are there. Perhaps the incentives promised by the earlier semantic
web advocates (search engine optimisation, the development of clients that
join up data between websites, etc.) were not strong enough for publishers,
but the need to control how that content is summarised and discovered in
places outside of the primary website is indeed a strong incentive.

There is much to be said for the argument that any enterprise that therefore
creates applied features to make use of content metadata (promoting related
content being just one example), that the organisation might create appropriate
pressure internally for those publishers to "correct" or at least improve
their semantic annotations when faced with clear evidence where improvements
could be made and errors corrected.

\section{Suitability for Semantics and Data Mining in Media Organisations}



\section{Suggestions for Initial Implementations in the Enterprise}

\section{Future Research}
\subsection{Better Enrichment}
\subsection{Patterns for Reducing Data Noise}
\subsection{Deeper Study of Machine Learning on Semantics}

