\chapter{Conclusions and Future Work}

This chapter lays out some conclusions from this research around the potential
for semantics being used more in the enterprise, whether semantics combined
with machine learning or data mining has potential in industry and how
organisations should consider implementing this. Finally, some options and
potential for future research are suggested.

\section{Challenges and Opportunitues for Semantics in the Enterprise}

As discussed in section~\ref{sec:production-recommendations} and also more
generally throughout chapter~\ref{chp:analysis}, there is much scope for
enterprises to benefit from using semantics in public-facing pages that display
their own content. This is particulary necessary in larger organisations where
attemping to gain an overview of all the content metadata in a short timeframe
would be a large enterprise integration effort.

The challenges amount to the issues raised around noise (semantic information
not immediately relevant to the content), incomplete information (due to
the open world assumption) and perhaps even erroneous information (embedded
Semantic Web data is not always visible enough for publishers to spot errors.)

Even with these issues, the experiment performed in this research was still
able to take place across a wide range of BBC content without any bespoke
integration against internal data APIs nor a any significant code created
to deal with a particular data source.

The power to use semantics and other web standards to gain a broad overview
of a large set of content should be very appealing to a media organisation,
even if there are issues with that data, assuming their application is in
a use case where less precision of that information is tolerable.

Even where data needed corrected, the software developed even in this
experiment was able to adapt to broad patterns in the data that were
problematic. there is also much indication that engineers in the enterprise
would be to complement the breadth from extracting semantics with the depth
of data obtained directly from data sources. This is achievable with the
set-theoretic nature of RDF graphs and our ability to perform simple set
union operations to join data sources.

As is evidenced by the pressure from social media and networks such as
Facebook and Twitter, publishers will include better semantics if the right
incentives are there. Perhaps the incentives promised by the earlier semantic
web advocates (search engine optimisation, the development of clients that
join up data between websites, etc.) were not strong enough for publishers,
but the need to control how that content is summarised and discovered in
places outside of the primary website is indeed a strong incentive.

There is much to be said for the argument that any enterprise that therefore
creates applied features to make use of content metadata (promoting related
content being just one example), that the organisation might create appropriate
pressure internally for those publishers to "correct" or at least improve
their semantic annotations when faced with clear evidence where improvements
could be made and errors corrected.

\section{Suitability for Semantics and Data Mining in Media Organisations}

As noted in section~\ref{sec:impact-of-noise}, the fuzzy and noisy nature
of linked data and the semantic web might find an ideal match in the world
of machine learning -- something that an operate on noisy and fuzzy data sets
and still find patterns and insight. That machine learning and data mining
builds upon the idea that data collected might be noisy or incomplete leads
to a strong argument for its application being ideal for the RDF model of
the semantic web.

Statistical methods like machine learning are clearly more tolerant of errors
than more ``hard'' uses such as presenting concrete facts to users based on
embedded semantics or inferring some hierarchical structure for a navigation
feature. Even where the machine learning methods struggle in this particular
research, it does not prevent further research that better tunes the learning
methods used to produce even more useful results.

The question remaining from all of this for the media enterprise, is whether
this combination of semantics and data mining has real application that is
useful.

TODO: Build on survey results

\section{Suggestions for Initial Implementations in the Enterprise}

Section~\ref{sec:production-recommendations} covers some recoomedations
for real implementations of semantics for data mining an organisation's
content.

In summary, an organisation can deliver initial value quickly with embedded
semantics (e.g. RDFa) extraction as a general-purpose solution is simple
to implement. The perhaps na\"ive mapping presented in this research for
converting RDF to feature sets performs well enough if the gaps and noise
in the data have a tolerable effect on the outcomes.

Media organisations can iterate very quickly on this simple idea and find
approaches that work for them and at each iteration consider the following:

\begin{itemize}
\item Can we implement entity extraction to supplement the embedded semantics?
\item How do we best tune any entity extaction over time?
\end{itemize}

\section{Future Research}
\subsection{Better Enrichment}
\subsection{Patterns for Reducing Data Noise}
\subsection{Deeper Study of Machine Learning on Semantics}

