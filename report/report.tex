\documentclass[10pt,a4paper]{report}

\title{Improving content discovery through combining linked data and data mining techniques}

\author{Ross Fenning}

\usepackage[dvipsnames]{xcolor}
\usepackage{nag}
\usepackage{rotating}
\usepackage[autosize]{dot2texi}
\usepackage[Bjarne]{fncychap}
\usepackage{tikz}
\usetikzlibrary{shapes,arrows,decorations}
\usepackage{listings}
\usepackage{textcomp}
\usepackage{mathtools}
\usepackage{caption}

\DeclareCaptionFont{white}{ \color{white} }
\DeclareCaptionFormat{listing}{
  \colorbox[cmyk]{0.43, 0.35, 0.35,0.01 }{
    \parbox{\textwidth}{\hspace{15pt}#1#2#3}
  }
}
\captionsetup[lstlisting]{ format=listing, labelfont=white, textfont=white, singlelinecheck=false, margin=0pt, font={bf,footnotesize} }

% Language Definitions for SPARQL
\lstdefinelanguage{sparql}{
morestring=[b][\color{blue}]\",
morekeywords={SELECT,CONSTRUCT,DESCRIBE,ASK,WHERE,FROM,NAMED,PREFIX,BASE,OPTIONAL,FILTER,GRAPH,LIMIT,OFFSET,SERVICE,UNION,EXISTS,NOT,BINDINGS,MINUS,a},
sensitive=true
}

\colorlet{punct}{red!60!black}
\definecolor{background}{HTML}{EEEEEE}
\definecolor{delim}{RGB}{20,105,176}
\colorlet{numb}{magenta!60!black}

\lstdefinelanguage{json}{
    basicstyle=\normalfont\ttfamily\footnotesize,
    numbersep=8pt,
    showstringspaces=false,
    breaklines=true,
    literate=
     *{0}{{{\color{numb}0}}}{1}
      {1}{{{\color{numb}1}}}{1}
      {2}{{{\color{numb}2}}}{1}
      {3}{{{\color{numb}3}}}{1}
      {4}{{{\color{numb}4}}}{1}
      {5}{{{\color{numb}5}}}{1}
      {6}{{{\color{numb}6}}}{1}
      {7}{{{\color{numb}7}}}{1}
      {8}{{{\color{numb}8}}}{1}
      {9}{{{\color{numb}9}}}{1}
      {:}{{{\color{punct}{:}}}}{1}
      {,}{{{\color{punct}{,}}}}{1}
      {\{}{{{\color{delim}{\{}}}}{1}
      {\}}{{{\color{delim}{\}}}}}{1}
      {[}{{{\color{delim}{[}}}}{1}
      {]}{{{\color{delim}{]}}}}{1},
}

\lstset{frame=single,captionpos=b}

\begin{document}

\maketitle
\tableofcontents

\chapter{Introduction}

Media companies produce ever larger numbers of articles, videos, podcasts,
games, etc. -- commonly collectively known as ``content''. A successful
content-producing website not only has to develop systems to aid producing and
publishing that content, but there are also demands to engineer effective
mechanisms to aid consumers in finding that content.

Approaches used in industry include providing a text-based search, hierarchical
categorisation (and thus navigation thereof) and even more tailored recommended
content based on past behaviour or content enjoyed by friends (or sometimes
simply other consumers who share your preferences).

\section{Problems}

There are several technical and conceptual problems with building effective
content discovery mechanisms, including:

\begin{itemize}

\item Large organisations can have content across multiple content management
systems, in differing formats and data models. Organisations face a large-scale
enterprise integration problem simply trying to gain a holistic view of all
their content.

\item Many content items are in fairly opaque formats, e.g. video content may be
stored as audio-visual binary data with minimal metadata to display on a
containing web page. Video content producers may not be motivated to provide
data attributes that might ultimately be most useful in determining if a user
will enjoy the video.

\item Content is being published continuously, which means any search or
discovery system needs to keep up with content as it is published and process it
into the appropriate data structures. Any machine learning previously performed
on the data set may need to be re-run.

\end{itemize}

\section{Hypothesis}

The following hypotheses are proposed for gaining new insights about an
organisation's diverse corpus of content:

\begin{itemize}

\item Research and software tools around the concept of \emph{Linked Data} can
aid us in rapidly acquiring a broad view (perhaps at the expense of depth) of an
organisation's content whilst also providing a platform for simple enrichment of
that content's metadata.

\item We can establish at least a na\"ive mapping of an RDF graph representing a
content item to an attribute set suitable for data mining. With such a mapping,
we can explore applying machine learning -- particularly unsupervised learning
-- across an organisation's whole content corpus.

\item Linked Data and Semantic Web \emph{ontologies} and models available can
provide data enrichment beyond attributes and keywords explicitly avaiable
within content data or metadata.

\item We can adapt established machine learning approaches such as clustering
for data published continuously in real time.

\item Many content-producers currently enrich their web pages with small
amounts of semantic metadata to provide better presentation of that content
as it is shared on social media. This enables simple collection of a full
breadth of content with significantly less effort than direct integration
with content management systems.

\end{itemize}

\chapter{Background}

This chapter discusses some of the existing research and technologies around
machine learning, RDF and combining them. It also covers some of the advantages
of using linked data and RDF in an enterprise setting and what tools and
approaches are well-defined enough that a corporation could build on top of
them rapidly.

Data mining activities such as machine learning rely on structuring data as
\emph{feature sets}\cite{bishop2006pattern} -- a set or vector of properties or
attributes that describe a single entity.
The process of \emph{feature extraction}
generates such feature sets from raw data and is a necessary early phase for
many machine learning activities.

The rest of this chapter will show:

\begin{enumerate}
\item that extracting feature sets from RDF\footnote{http://www.w3.org/TR/PR-rdf-syntax/} graphs can be done elegantly and follows naturally from some
previous work in this area; and
\item that the RDF graph is a suitable and even desirable data model for content
metadata in terms of acquiring, enriching and even transforming that data ahead
of feature extraction.
\end{enumerate}

\section{Data Mining}

TODO

\section{RDF and Feature Extraction}
\label{sec:rdf-and-features}

The RDF graph is a powerful model
for metadata based on representing knowledge as a set of
subject-predicate-object \emph{triples}. The query language, SPARQL, gives us a
way to query the RDF graph structure using a declarative pattern and return a
set of all variable bindings that satisfy that pattern.

For example, the SPARQL query in Listings~\ref{lst:sparqlfoaf}
queries an RDF graph that contains contact information and returns the
names and email address of all ``Person'' entities therein.

Notably, Kiefer, Bernstein and Locher\cite{kiefer2008adding} proposed a novel
approach called SPARQL-ML -- an extension to the
SPARQL\cite{segaran2009programming} query language with new keywords to
facilitate both generating and applying models. This means that the system
capable of parsing and running standard queries must also run machine learning
algorithms.

Their work involved developing an extension to the SPARQL query
engine for \emph{Apache Jena}\footnote{https://jena.apache.org/} that integrates
with systems such as \emph{Weka}\footnote{http://www.cs.waikato.ac.nz/ml/weka/}.
A more suitable software application for enterprise use might focus solely on
converting RDF graphs into a neutral data structure that can plug into arbitrary
data mining algorithms.

\begin{lstlisting}[label=lst:sparqlfoaf,caption={Example SPARQL query for people's names and email addresses},language=sparql]
PREFIX foaf: <http://xmlns.com/foaf/0.1/>
SELECT ?name ?email
WHERE {
  ?person a foaf:Person.
  ?person foaf:name ?name.
  ?person foaf:mbox ?email.
}
\end{lstlisting}

If we consider an RDF graph, $g$, to be expressed as a set of triples:

\begin{displaymath}
  (s, p, o) \in g
\end{displaymath}

\noindent this query could then be
expressed as function $f: G \rightarrow (S \times S)$ where $G$ is the set of
all possible RDF graphs and $S$ is a set of all possible strings.
This allows the result of the
SPARQL query to be expressed as a set of all SELECT variable bindings that
satisfy the WHERE clause:

$$
q(g,n,e) = \exists p . (p, type, Person) \in g\ \land (p, name, n) \in g \land (p, mbox, e) \in g
$$

$$
g \in G \models f(g) = \{(n, e) \subseteq (S \times S) \: | \: q(g,n,e)\}
$$

This could be generalised to express a given feature set as
vector $(a_1, a_2, ..., a_n)$:

$$
g \in G \models (a_1, a_2, ..., a_n) \in f(g)
$$

\noindent and in the case where all $a_k \in f(g)$ are literal (e.g. string or
numeric) values, we can thus consider a given SPARQL query to be specific
function capable of feature extraction from any RDF graph into sets of
categorical or numeric features.

\begin{lstlisting}[label=lst:sparqlabout,caption={SPARQL query to determine what },language=sparql]
PREFIX rdf: <http://www.w3.org/1999/02/22-rdf-syntax-ns#>
SELECT ?topic
WHERE {
  ?article rdf:about ?topic .
}
\end{lstlisting}

This might allow a query that extracts a country's population, GDP, etc.
provide feature extraction for learning patterns in economics, for example.
However, this is limited to features derived from single-valued predicates
with literal-valued ranges. It is not clear how to formulate a query that
expresses whether or not a content item is about a given topic.

In the RDF
model, it would be more appropriate to use a query like that in
Listings~\ref{lst:sparqlabout} where for a given $?article$ identified by
URI, we can get a list of URIs identifying concepts which the article mentions.
Such a query might be expressed as function $f': G \rightarrow \mathcal P(U)$ where $U$ is
set of all URIs such that:

$$
g \in G \models f'(g, uri) = \{t \: | \: (uri, about, t) \in g\} 
$$

An approach of generating attributes for a given resource was proposed by
Paulheim and F\"urnkranz\cite{paulheim2012unsupervised}. They defined specific
SPARQL queries and provided case study evidence for the effectiveness of
each strategy.

Their work focused on starting with relational-style data (e.g. from a
relational database) and using \emph{Linked Open Data} to identify entities
within literal values in those relations and generated attributes from
SPARQL queries over those entities.

For a large content-producer, there is a more general problem where many content
items do not have a relational representation and the content source is a body
of text or even a raw HTML page. However, the feature generation from Paulheim
and F\:urnkranz proves to be a promising strategy given we can acquire an RDF
graph model for content items in the first place.

\section{RDF in the enterprise}
\label{sec:linked-enterprise-data}

TODO

\chapter{System Design}

In this chapter, a system is inductively derived and concretely design to make
use of multiple strategies for:

\begin{enumerate}
\item gathering (meta)data about all of an organisations content items;
\item extracting metadata not explicitly modelled in source content management
systems;
\item further enriching that metadata with information not explicitly present
in the content item itself; and
\item applying machine learning to that content metadata to gain new insights
about that content.
\end{enumerate}

Initially, a business context is described to produce a design for a system
that could be a applied within a media or content-producing organisation. This
context will guide all design decisions.

\section{Context}

%TODO

\section{Use Cases}

%TODO

\section{Technical Architecture}

%TODO

\section{Data Pipeline}

A core subsystem in the overall system is a conceptual data pipeline whose
input is a URI or IRI identifying a content item published on an organisation's
website and the output is feature sets ready for applying machine learning.

In this section, a theoretical pipeline is inductively defined in steps such
that an application of this pipeline would choose to implement some subset of
all potential pipeline stages as appropriate for the relevant problem domain.

In Chapter~\ref{chp:implementation}, aa system is engineered that implements
as many of these pipeline stages as possible such that a running instance of
the application can configure which components to use and which not to use.
Then in Chapter~\ref{chp:evaluation}, an evalution of the system is given
while it is running each component in isolation to demonstrate which of the
theoretically-defined processes in this chapter appears to be most effective
in generating feature sets specifically for clustering web content.

\subsection{Definitions}

This system requires some initial definition of some data structures in use:

\begin{description}

\item[IRI] \hfill \\
The input to the system is a character string conformant to the
IRI syntax defined in RFC 3987\footnote{http://tools.ietf.org/html/rfc3987}.
This allows more generality offered by
URIs\footnote{http://tools.ietf.org/html/rfc3986} but is trivially made
compatible with systems that use URIs through the conversion algorithm defined
in section~3.2 of RFC 3987. Note that the public URL by which the public can
read or otherwise consume the content is a valid identifier, but we are not
restricted to that.

\item[Feature Set] \hfill \\
The final output of this data pipeline is a data structure
analogous to a relation or tuple per IRI fed into the system. Every IRI should
have a literal value against all possible columns or fields. For binary fields,
(e.g. the presence of absence of a concept tag), a more pragmatic structure
might be a list of tags positively associated with the IRI rather than
explicitly assigning $false$ to all tags to which the content item does not
pertain. This is analogous to a spare matrix when dealing with a large number
of dimensions.

\item[Named RDF Graph] \hfill \\
The structure used throughout most of the data pipeline
is that of an RDF graph. This is used for all the benefits outlined in
Section~\ref{sec:linked-enterprise-data} such as ease of transformation and
combining of data sets. Named graphs are used such that all data acquired
are keyed back to the IRI of the content item being processed. This also allows
all graphs to be combined in a \emph{triplestore} if needed to allow SPARQL
queries across the combined data for all content items. This can be modelled as
a data structure in many programming languages, but where a serialisation is
used (e.g. examples shown here or to send the data between components), the
JSON-LD\cite{sporny2014json} syntax will be used.
\end{description}

\subsection{Identity Graph}

\begin{centering}
\begin{lstlisting}[label=lst:jsonld-identity,caption={Identity graph for a content item in JSON-LD syntax},language=json]
{
  "@id": "http://example.com/entity/1",
  "@graph": []
}
\end{lstlisting}
\end{centering}

With the knowledge only of a content item's IRI, we are arguably only able to
produce an empty named RDF graph. Such a graph for an example IRI
\texttt{http://example.com/entity/} is illustrated in
JSON-LD syntax in Listings~\ref{lst:jsonld-identity}.

The most na\"ive feature set we can generate from such an RDF graph is clearly
a singleton relation \texttt{("http://example.com/entity/")} where a single
$IRI$ field has the value \texttt{"http://example.com/entity/"}. It is also
clear that a set of one-dimension feature vectors with unique values in each
is not suitable for any form of machine learning activity. This does, however,
illustrate a baseline for a working software application that is -- at least
in the syntactic sense -- transforming IRI inputs to feature sets outputs.
Such a $null$ feature generator is depicted in Figure~\ref{fig:gen-null}.

\begin{figure}[h]
  \begin{center}
    \begin{dot2tex}[dot,options=-t math,autosize,pgf,scale=0.7]
      digraph g {
        rankdir=LR;

        node [shape=circle,margin="0,0"];
        edge [len=2];

        IRI -> RDF [label="extract"];
        RDF -> Features [label="(IRI)"];
      }
    \end{dot2tex}
  \end{center}
  \caption{Null feature generator \label{fig:gen-null}}
\end{figure}

Note that Figure~\ref{fig:gen-null} shows all three data structures involved
despite having no functional use. We can also see top-level definitions of the
process where we first \emph{extract} semantic information in RDF from a
content item indentified by IRI and then \emph{generate} features therefrom.
More useful models can now be inductively defined by adding atomic
subcomponents that may each add value to the overall transformation.

There are three clear axes along which we can improve this pipeline:
\emph{extract} more RDF data knowing only an item's IRI,
expand or \emph{enrich} an existing RDF graph
and then improve how we \emph{generate} features for data mining.
In the first instance, we can consider the former and add
a single pipeline stage for expanding the RDF graph.

\subsection{RDF Extraction}
\label{sec:rdf-extraction}

Tim Berners-Lee outlined four rules\cite{berners2011linked} for Linked Data,
rule number three of which states ``When someone looks up a URI, provide useful
information, using the standards''. If we assume that many pages have embedded
some semantic web or RDF data, then a simple extraction strategy would be
to deference the content item's IRI via an HTTP GET and pass the content
to a parser capable of extracting RDFa, microformats, etc.

Many tools such as the RDFLib\footnote{https://github.com/RDFLib/rdflib}
provide functionality for taking a URL and returning an RDF graph of all data
found when fetching the resource it represents, so this is arguably an ideal
first choice in attempting to learn something about a content item from its
IRI. 

\begin{figure}[h]
  \begin{center}
    \begin{dot2tex}[dot,options=-t math,autosize,pgf,scale=0.7]
      digraph g {
        rankdir=LR;

        node [shape=circle,margin="0,0"]

        IRI -> RDFa [label="derefernce"];
        RDFa -> RDF [label="parse"];
        RDF -> Features [label="(IRI)"];
      }
    \end{dot2tex}
  \end{center}
  \caption{Semantic web data extraction\label{fig:gen-rdfa}}
\end{figure}

Figure~\ref{fig:gen-rdfa} depicts the pipeline with a simple dereference
step added. Note that the feature set generated is still the singleton
relation with only the IRI value. Thus the next step should be to add a step
that improves the feature set generation.

\subsection{Feature Set Generation}

Paulheim and F\"urnkranz\cite{paulheim2012unsupervised} described a number of
SPARQL queries for generating feature sets from RDF data, which could inspire
a simple query such as that shown in Listings~\ref{lst:simple-sparql}. This
query generates a boolean \texttt{true} value for any properties that match
and implies \texttt{false} for those that do not.

\begin{lstlisting}[label=lst:simple-sparql,caption={Generates field \texttt{content\_?p\_?v} with value \texttt{true}},language=sparql]
SELECT ?p ?v
WHERE { ?iri ?p ?v . }
\end{lstlisting}

As also noted by Paulheim
and F\"urnkranz, this overlooks the
\emph{open world assumption}\cite{russel2010artificial}. However, application
of clustering algorithms on binary data can employ asymmetric distance metrics
such as Jaccard similarity coeffcient\cite{witten2005data}, which notably
avoids deriving similarity from negative values. That is, two content items
lacking a particular property will contribute no information about their
(dis)similarity. Thus we safely avoid inadvertently grouping together one item
that genuinely lacks the property with another that indeed has the property,
but we lack the positive assertion thereof in the data extracted.

\begin{figure}[h]
  \begin{center}
    \begin{dot2tex}[dot,options=-t math,autosize,pgf,scale=0.7]
      digraph g {
        rankdir=LR;

        node [shape=circle,margin="0,0"]

        IRI -> RDFa [label="derefernce"];
        RDFa -> RDF [label="parse"];
        RDF -> Features [label="content\_?p\_?v"];
      }
    \end{dot2tex}
  \end{center}
  \caption{Semantic web content extraction with basic SPARQL feature generation\label{fig:gen-rdfa-basic}}
\end{figure}

The basic pipeline in Figure~\ref{fig:gen-rdfa} can thus be augmented with this
basic feature extraction to produce the pipeline depicated in
Figure~\ref{fig:gen-rdfa-basic}.

\subsection{RDF Enrichment}

The third and final direction in which this data pipeline can be improved is
in terms of data enrichment. A simple strategy here is to repeat the
deferencing used in Section~\ref{sec:rdf-extraction}, but for each IRI
found as the object of a triple in which the initial IRI is the subject.
Formally:

$$
g \in G \models \exists p, o . (IRI, p, o) \in g \rightarrow g' = deref(o)
$$

That RDF graphs can be modelled as mathematical sets as in
Section~\ref{sec:rdf-and-features} means we can express a graph enriched this
way as a \emph{union} of the initial graph with each graph returned from all
dereferencing:

$$
g \in G \models g' = g \cup \bigcup \: \{deref(o) \: | \: (IRI, p, o) \in g\}
$$

Figure~\ref{fig:object-deref} shows the data pipeline with this additional
enrichment stage. Note that now we have potential for some stages being
executed in parallel.

\begin{sidewaysfigure}[h]
  \begin{center}
    \begin{dot2tex}[dot,options=-t math,autosize,pgf,scale=0.7]
      digraph g {
        rankdir=LR;

        node [shape=circle,margin="0,0"];

        dummy [shape=none,label=""];
        
        IRI [label="IRI"];
        RDF1 [label="RDF_1"];
        RDF2 [label="RDF_2"];
        RDFp [label="RDF'"];
        
        IRI -> RDFa [label="dereference"];
        RDFa -> RDF [label="parse"];
        RDF -> RDF1 [label="dereference/parse"];
        RDF -> RDF2 [label="dereference/parse"];
        RDF -> RDFp [label="\cup"];
        RDF1 -> RDFp [label="\cup"];
        RDF2 -> RDFp [label="\cup"];
        RDFp -> Features [label="content\_?p\_?v"];
      }
    \end{dot2tex}
  \end{center}
  \caption{\label{fig:object.-deref}Semantic web content miner with addtional dereferencing of linked entities}
\end{sidewaysfigure}

So far in this section, we have inductively built up a data pipeline from
a ``null'' base working with only identity graphs to a simple pipeline
capable of \emph{extracting} an RDF graph, \emph{enriching} it and then
\emph{generating} features from it.

What needs to be proven through experimentation is \emph{which} of these
provides the information required for effective data mining. As part of this
experimentation, we can now look at further techniques and approaches to try.

\subsection{Improving Extraction}

In order to gain a larger set of RDF data at the start of the pipeline, we
can derive further ways to get information about a content item given only
its IRI as input.

Rizzo and Troncy\cite{rizzo2012nerd} defined a framework called NERD capable
of combining multiple entity extraction systems to provide a unified way of
identifying -- and disambiguating -- named entities within a given body of text.
With such a system, we can create a second, parallel RDF extraction strategy
that creates a graph of triples in the form:

$$
(IRI, rdf\!\!:\!\!about, Entity)
$$

\noindent where $Entity$ is an IRI representing a concept or entity believed
to be found in the content's textual content. A data pipeline complementing
the RDFa-based extraction is depicted in Figure~\ref{fig:entity-extraction}.
Note the ability to apply a simple set union to the result of each extraction
as with the enrichment.

\begin{sidewaysfigure}[h]
  \begin{center}
    \begin{dot2tex}[dot,options=-t math,autosize,pgf,scale=0.8]
      digraph g {
        rankdir=LR;

        node [shape=circle,margin="0,0"]
        RDF1 [label="RDF_1"];
        RDF2 [label="RDF_2"];

        IRI -> RDFa [label="derefernce"];
        IRI -> Text [label="GET"];
        RDFa -> RDF1 [label="parse"];
        Text -> RDF2 [label="entity extraction"];
        RDF1 -> RDF [label="\cup"];
        RDF2 -> RDF [label="\cup"];
        RDF -> Features [label="content\_?p\_?v"];
      }
    \end{dot2tex}
  \end{center}
  \caption{Named entity extraction in addition to semantic web extraction\label{fig:entity-extraction}}
\end{sidewaysfigure}

Another strategy can be to infer a relationship between two content items where
one contents an HTML link to another. It is not always possible to derive
precise semantics of such a link (unless the publisher has kindly provided
a \texttt{rel} attribute), but a weak relationship such as:

$$
(IRI_1, ex:related, IRI_2)
$$

\noindent might prove -- through experiementation -- to be useful enough
for data mining insights.

The fourth and final extraction strategy explored is acquiring metadata from
bespoke Content Management Systems and other internal APIs. This is generally
the only option used in enterprises settings, as discussed in
Section~\ref{sec:linked-enterprise-data}. This is likely to be the richest
source of information where an enterprise has typically preferred bespoke
integrations against non-hypermedia interfaces, so experimentation should
help quantity or qualify the value such a direct integration adds (perhaps to
consider it in combination with the cost of bepsoke, repeated integration
projects).

The assertion explored here is that such bespoke integrations can
\emph{complement} cheaper work such as extracting RDFa with pre-built tools
(and thus be developed one-by-one after the initial release of an application
such as this data pipeline). Note that repeated custom integration projects
means that each data source requires a different application be developed (as
opposed to the reuse of a single RDFa parser or HTML link scraper). This
also means we are not necessarily comparing like-for-like if we introduce only
one at a time. It also makes it
difficult to evaluate data mining of a diverse content corpus if an integration
against an API provides additional metadata for only, say, 10\% of that corpus.

These challenges aside, it is clear that bespoke integrations have a clear place
in this data pipeline being applied in a real enterprise setting. Now that
we have a complete set of theoretical stages for \emph{extraction}, the
remaining improves lie now in the \emph{enrichment} and
\emph{feature generation} stages.

\subsection{Improving Enrichment}

In addition to enriching through dereferencing linked entities, it is proposed
to explore the following options:

\begin{itemize}
\item inferring relationships to hypernyms as defined by Wordnet\cite{miller1995wordnet};
\item inferring facts based using rules derirved from expert domain knowledge;
\item using RDFS and OWL to generate new triples with well-established Ontology rules.
\end{itemize}

In the first approach, we can consider a relationship rule such as:

$$
(IRI, ex:related, ex:Dog)
$$

\noindent and \emph{infer} the fact:

$$
(IRI, ex:related, ex:Animal)
$$

\noindent and produce an enriched graph containing all additional facts
inferred in this way.

When dealing with proper nouns and named entities, inferring facts based on
domain knowledge may be more appropriate. For instance, the rule in n3 syntax:

\begin{lstlisting}
  {
    ?article ex:takesPlaceIn ?city .
    ?city a ex:City .
    ?city ex:capitalOf ?country .
  } -> { ?iri ex:takesPlaceIn ?country }
\end{lstlisting}

\noindent might be useful ot help cluster articles that take place in the same
country -- even if the countries are not always explicitly mentioned therein.
Such an inference requires knowledge about cities and countries to
write and domain experts for different types of content might be able to offer
more nuanced rules.

An example for BBC content might be to infer that all articles written under
the \emph{Newsround} brand is suitable for children or that programmes that
have broadcast times during the day are also suitable for children.

The third and final proposed improvement makes use of standard tools to find
\emph{closures} using, e.g. RDFS, ontology rules. With this approach, we
can infer that entities that have a given type or class also have their
superclasses and supertypes. This gives us similar inference to hypernyms,
but with knowledge present in well-established ontologies.

An obvious example might where two content items have been identified as
related to the same concept -- so they would be candidates for clustering
together -- but when RDF data are extracted, it is found that two
different URIs have been used for each:

\begin{displaymath}
(IRI_{1}, <\!\!http\!:\!\!//dbpedia.org/property/related\!\!>, ex1:entity)\\
(IRI_{2}, <\!\!http\!:\!\!//dbpedia.org/property/related\!\!>, ex2:anotherEntity)
\end{displaymath}

In the RDF graph for $IRI_2$, say, we might find the source had provided an
\texttt{owl:sameAs} assertion such as:

\begin{displaymath}
(ex2:anotherEntity, owl:sameAs, ex1:entity)
\end{displaymath}

This is possible in the case where the second item's data source uses its own
set of identifiers for entities, but has chosen to provide equivalences to
a more standard set of identifiers (e.g. DBpedia). With this information,
our data pipeline can infer:

\begin{displaymath}
(IRI_{1}, <\!\!http\!:\!\!//dbpedia.org/property/related\!\!>, ex1:entity)\\
(IRI_{2}, <\!\!http\!:\!\!//dbpedia.org/property/related\!\!>, ex2:anotherEntity)\\
(IRI_{2}, <\!\!http\!:\!\!//dbpedia.org/property/related\!\!>, ex1:entity)
\end{displaymath}

\noindent and the feature generation stage might provide the common features
\texttt{dbprop\_related\_ex1\_entity=true} for both content items.

\subsection{Improving Feature Generation}

The feature generation outlined so far relies solely on boolean values
indicating whether or not a given content item is related by some property to
some entity. This recreates the concept of \emph{tagging} where a given
object is either associated or not associated with a series of \emph{tags}.

One of the advantages of the RDF graph model is that we are not constrained
necessarily to properties and attributes directly applicable to the entity. We
could imagine adding a level of indirection to the query in
Listings~\ref{simple-sparql} to create the query in
Listings~\ref{level2-sparql}.


\begin{lstlisting}[label=lst:level2-sparql,caption={Generates field \texttt{content\_?p1\_?p2\_?v} with value \texttt{true}},language=sparql]
SELECT ?p1 ?p2 ?v
WHERE {
  ?iri ?p1 ?o .
  ?o ?p2 ?v .
}
\end{lstlisting}

With the this query, features of the form \texttt{content\_?p1\_?p2\_?v} can
be generated. An example of this might be where a television programme
content item has information about the actors that appeared therein, e.g.
\texttt{ex:hasActor}, and furthermore we have information about those actors
such as where they were born, e.g. \texttt{ex:bornIn}. With the path created
by following both of these predicates, it is possible to create features for
a television programme such as
\texttt{content\_ex:hasActor\_ex:bornIn\_Edinburgh} and we can potentially
find similarity between programmes where the actors were born in the same
city.

Perhaps a third step in the predicate path followed can give us even more
useful features. The example above could be expanded to
\texttt{content\_ex:hasActor\_ex:bornIn\_ex:cityIn\_Scotland} to allow the
more general ability to cluster programmes with Scottish actors, for instance.

Appropriate experimentation should show whether more value is gained by
adding these additional levels of indirection.

\subsection{Maximal Data Pipeline}

In this section, a data pipeline was inductively built up from a base,
identify pipeline with suggestions for potential improvements in different
stages. An application of all the ideas discussed so far might look like
that depicated in Figure\ref{fig:maximal-pipeline}.

\begin{sidewaysfigure}[p]
  \begin{center}
    \begin{dot2tex}[dot,options=-t math,autosize,pgf]
      digraph g {
        rankdir=LR;

        node [shape=circle,margin="0,0"]
        RDF1 [label="RDF_1"];
        RDF2 [label="RDF_2"];
        RDF3 [label="RDF_3"];
        RDFp [label="RDF'"];
        RDFp1 [label="RDF'_1"];
        RDFp2 [label="RDF'_2"];
        RDFp3 [label="RDF'_3"];
        RDFp4 [label="RDF'_4"];

        IRI -> RDF1 [label="\text{parse}"];
        IRI -> RDF2 [label="\text{extract}"];
        IRI -> RDF3 [label="\text{scrape}"];

        RDF1 -> RDF [label="\cup"];
        RDF2 -> RDF [label="\cup"];
        RDF3 -> RDF [label="\cup"];

        RDF -> RDFp1 [label="\text{dereference/parse}"];
        RDF -> RDFp2 [label="\text{hypernym inference}"];
        RDF -> RDFp3 [label="\text{expert inference}"];
        RDF -> RDFp4 [label="\text{OWL inference}"];

        RDF -> RDFp [label="\cup"];
        RDFp1 -> RDFp [label="\cup"];
        RDFp2 -> RDFp [label="\cup"];
        RDFp3 -> RDFp [label="\cup"];
        RDFp4 -> RDFp [label="\cup"];
        
        RDFp -> Features [label="generate"];
      }
    \end{dot2tex}
  \end{center}
  \caption{\label{fig:maximal-pipeline}Maximal Data Pipeline}
\end{sidewaysfigure}

\section{Pipeline Architecture}


\chapter{Implementation}
\label{chp:implementation}

\chapter{Evaluation}
\label{chp:evaluation}

\chapter{Conclusion}

\bibliographystyle{plain}
\bibliography{references}

\end{document}
