\documentclass[10pt,a4paper]{report}
\usepackage[dvipsnames]{xcolor}
\usepackage{nag}
\usepackage{rotating}
\usepackage[autosize]{dot2texi}
\usepackage[Bjarne]{fncychap}
\usepackage{tikz}
\usetikzlibrary{shapes,arrows,decorations}
\usepackage{listings}
\usepackage{textcomp}
\usepackage{mathtools}
\usepackage{caption}
\usepackage{rotating}
\usepackage{filecontents}
\usepackage{pgfplots, pgfplotstable}
\usepgfplotslibrary{statistics}

\DeclareCaptionFont{white}{ \color{white} }
\DeclareCaptionFormat{listing}{
  \colorbox[cmyk]{0.43, 0.35, 0.35,0.01 }{
    \parbox{\textwidth}{#1#2#3}
  }
}
\captionsetup[lstlisting]{ format=listing, labelfont=white, textfont=white, singlelinecheck=false, margin=0pt, font={bf,footnotesize} }

% Language Definitions for SPARQL
\lstdefinelanguage{sparql}{
morestring=[b][\color{blue}]\",
morekeywords={SELECT,CONSTRUCT,DESCRIBE,ASK,WHERE,FROM,NAMED,PREFIX,BASE,OPTIONAL,FILTER,GRAPH,LIMIT,OFFSET,SERVICE,UNION,EXISTS,NOT,BINDINGS,MINUS,a},
sensitive=true
}

\colorlet{punct}{red!60!black}
\definecolor{background}{HTML}{EEEEEE}
\definecolor{delim}{RGB}{20,105,176}
\colorlet{numb}{magenta!60!black}

\lstdefinelanguage{json}{
    basicstyle=\normalfont\ttfamily\footnotesize,
    numbersep=8pt,
    showstringspaces=false,
    breaklines=true,
    literate=
     *{0}{{{\color{numb}0}}}{1}
      {1}{{{\color{numb}1}}}{1}
      {2}{{{\color{numb}2}}}{1}
      {3}{{{\color{numb}3}}}{1}
      {4}{{{\color{numb}4}}}{1}
      {5}{{{\color{numb}5}}}{1}
      {6}{{{\color{numb}6}}}{1}
      {7}{{{\color{numb}7}}}{1}
      {8}{{{\color{numb}8}}}{1}
      {9}{{{\color{numb}9}}}{1}
      {:}{{{\color{punct}{:}}}}{1}
      {,}{{{\color{punct}{,}}}}{1}
      {\{}{{{\color{delim}{\{}}}}{1}
      {\}}{{{\color{delim}{\}}}}}{1}
      {[}{{{\color{delim}{[}}}}{1}
      {]}{{{\color{delim}{]}}}}{1},
}

\lstset{frame=single,captionpos=b}


\title{Improving content discovery through combining linked data and data mining techniques}
\author{\vfill\textbf{Ross Fenning}}
\affil{
\vfill
  Dissertation submitted in partial fulfillment of the
\\
requirements for the degree of
\\
Master by Advanced Study in Software Engineering and Internet Architecture

\vfill


\textbf{School of Electrical Engineering \& Computer Science}
\\
\textbf{University of Bradford}
\vfill
}

\begin{document}

\maketitle

\begin{abstract}
\vfill
\noindent MSc in Software Engineering and Internet Architecture 2016
\vfill
\noindent Improving content discovery through combining linked data and data mining techniques
\vfill
\noindent By Ross Fenning
\vfill
\noindent Project Supervisor: Dr. D. Thakker
\vfill
\noindent A comparative study of multiple approaches for extracting linked data
from web content published by the BBC for the purposes of applying
machine learning to cluster related content.

This project designs and implements a process by which semantic data
about any piece of online media content can be extracted into RDF
models in
multiple ways and feature
sets for machine learning can be generated from
that data. The clusters produced by fourteen different permutations
of techniques are evaluated and a potential application of suggesting
related content is mocked up for qualitative evaluation.

It is concluded organisations can very easily make use of
embedded semantics within web pages due to better semantic properties
being made available in recent years because of pressure from
social media sites such as Facebook and Twitter.
These embedded semantics provide a good baseline for simple
learning that can group content based on broad categorisation.

However, feedback from the evaluation suggests that real value for
users from suggesting related content comes additionally from linking items
based on topics and themes, which are better found via entity
extraction systems such as DBPedia Spotlight.

A simple application of entity extraction shows promising results in
evaluation and is highlighted as an area for organisations to explore
tuning to provide a powerful complement to embedded semantics.

\end{abstract}

\tableofcontents

\chapter{Introduction}

Media companies produce ever larger numbers of articles, videos, podcasts,
games, etc. -- commonly collectively known as ``content''. A successful
content-producing website not only has to develop systems to aid producing and
publishing that content, but there are also demands to engineer effective
mechanisms to aid consumers in finding that content.

Approaches used in industry include providing a text-based search, hierarchical
categorisation (and thus navigation thereof) and even more tailored recommended
content based on past behaviour or content enjoyed by friends (or sometimes
simply other consumers who share your preferences).

\section{Problems}

There are several technical and conceptual problems with building effective
content discovery mechanisms, including:

\begin{itemize}

\item Large organisations can have content across multiple content management
systems, in differing formats and data models. Organisations face a large-scale
enterprise integration problem simply trying to gain a holistic view of all
their content.

\item Many content items are in fairly opaque formats, e.g. video content may be
stored as audio-visual binary data with minimal metadata to display on a
containing web page. Video content producers may not be motivated to provide
data attributes that might ultimately be most useful in determining if a user
will enjoy the video.

\item Content is being published continuously, which means any search or
discovery system needs to keep up with content as it is published and process it
into the appropriate data structures. Any machine learning previously performed
on the data set may need to be re-run.

\end{itemize}

\section{Hypothesis}

The following hypotheses are proposed for gaining new insights about an
organisation's diverse corpus of content:

\begin{itemize}

\item Research and software tools around the concept of \emph{Linked Data} can
aid us in rapidly acquiring a broad view (perhaps at the expense of depth) of an
organisation's content whilst also providing a platform for simple enrichment of
that content's metadata.

\item We can establish at least a na\"ive mapping of an RDF graph representing a
content item to an attribute set suitable for data mining. With such a mapping,
we can explore applying machine learning -- particularly unsupervised learning
-- across an organisation's whole content corpus.

\item Linked Data and Semantic Web \emph{ontologies} and models available can
provide data enrichment beyond attributes and keywords explicitly avaiable
within content data or metadata.

\item We can adapt established machine learning approaches such as clustering
for data published continuously in real time.

\item Many content-producers currently enrich their web pages with small
amounts of semantic metadata to provide better presentation of that content
as it is shared on social media. This enables simple collection of a full
breadth of content with significantly less effort than direct integration
with content management systems.

\end{itemize}

\chapter{Background}

This chapter discusses some of the existing research and technologies
around machine learning, RDF and combining them. It also covers some
of the advantages of using linked data and RDF in an enterprise
setting and what tools and approaches are well-defined enough that a
corporation could build on top of them rapidly.

Data mining activities such as machine learning rely on structuring
data as \emph{feature sets}\cite{bishop2006pattern} -- a set or vector
of properties or attributes that describe a single entity.
The process of \emph{feature extraction} generates such feature sets
from raw data and is a necessary early phase for many machine learning
activities.

The rest of this chapter will show:

\begin{enumerate}
\item that extracting feature sets from
  RDF\footnote{http://www.w3.org/TR/PR-rdf-syntax/} graphs can be done
  elegantly and follows naturally from some previous work in this area; and
\item that the RDF graph is a suitable and even desirable data model for content
  metadata in terms of acquiring, enriching and even transforming that data ahead
  of feature extraction.
\end{enumerate}

\section{Data Mining}
\subsection{Overview}

\emph{Data Mining} is a process for finding knowledge and information
in data sets by drawing on research from artificial intelligence,
machine learning and statistics.\cite{han2011data} In this research,
the focus is on the machine learning aspect with
the data extraction phase implemented using semantics and RDF.

\subsection{Supervised Learning}

\emph{Machine Learning} broadly breaks down into \emph{supervised}
and \emph{unsupervised} learning. In supervised learning, we have
a task with a clear goal for which we attempt to create a computer
model that can perform that task. In this context, a task might be
to categorise an entity based on its attributed (e.g. diagnose
a particular illness based on symptoms and assessment data from a
patient exam) or perhaps a more numerical goal such as predicting
the chance of rain on a given day based on multiple numerical
readings.

Well-known examples include neural networks, decision trees and or
linear regression. In all supervised learning techniques, we can
normally assess the model's performance against training data for
which we already know the answers. We can see supervised learning
thus as a computational attempt to extrapolate human knowledge into
a reusable model that can predict or determine things at a speed
and scale humans would not normally be able to work through manually
(e.g. predicting weather with manual calculations or diagnosing
patients more quickly and with greater accuracy).

\subsection{Unsupervised Learning}

In unsupervised learning, there is not necessarily a specific goal
against which we can judge the models, but instead we are applying
more exploratory methods to tell us patterns in the data that maybe
we could not see for ourselves (e.g. if we would have to graph trends
over more than two or three dimensions, it is usually difficult to
visualise).

A well-known example is association rule mining that can find
interesting relations in data sets that were perhaps previously
unknown. A classic implementation of this technique is in
supermarkets and other commercial enterprises where those
organisations seek to find rules and patterns in consumer purchasing
habits so that they can try to promote items that customers might
want to buy based on other items they are currently trying to
purchase.

\subsection{Clustering}

Another major unsupervised learning technique -- and the focus of
this research -- is \emph{clustering}. In clustering algorithm, we
seek to group similar items or entities together in groups or
\emph{clusters}. Clustering has many applications, but in this
work the focus will be around \emph{hierarchical} clustering which
is effective for grouping together items that appear to be
connected in some way -- in our case that the media content is
connected by virtue of a common theme, topic or genre.

This is based on the premise that users that like a particular
content item might well be interested in other items that mention
the same topics, fall into the same genres or are in some other
way linked.

Note that association rule mining is arguably useful in a similar way.
In the case where we know from user behaviour that users like e.g.
content item $A$ also like content item $B$, then a new user that
likes $A$ but has not read or watched $B$ is likely to be interested
in $B$, so we can suggest it to them.

This is very powerful -- perhaps more powerful as it models how
people behave, not arbitrary tagging or classification of content --
but the focus of this research experiment is to explore the insights
we can get \emph{from the content data alone}. This is appealing
where content producers are struggling to understand their users'
behaviour as collecting meaningful analytics and records of user
behaviour is by no means a trivial endeavour in its own right.

Hierarchical clustering differs from, say,
\emph{k-means clustering}\cite{witten2005data} in that
we start with no assumption about the number of clusters or groups
the data will be divided into. This method is generally applied
\emph{agglomeratively} or \emph{divisively}.

In divisive, hierarchical clustering, we start with all elements in
a single cluster and proceed to find divisions that best split that
cluster. This is applied recursively until we generate a
dendrogram\cite{witten2005data} showing a full hierarchy of similarity
between the items or we can halt the division based on some condition,
e.g. only split clusters with a low cohesion below some threshold
and stop on the first attempt to split a cohesive cluster above
that threshold.

Agglomerative, hierarchical clustering goes in the opposite direction
by starting with all items in their own, individual clusters.
The two ``nearest'' clusters are then recursively merged until, again,
we have a full dendrogram or stop before we are about to create a
cluster whose cohesion falls below some threshold.

\subsection{Distance and Linkage}

Clustering relies on grouping items that seem close, which creates
a necessity to define what is meant by close. The distance between
any two items is normally determined by a distance
function.\cite{witten2005data} Such a function will normally take
two items and return a value between 0 (items are the same) up to an
an unbounded value as the items are considering further and further
apart.

Related to distance and specific to clustering is the concept of
linkage, which describes the distance between two clusters. A linkage
function will make use of a distance function. Three examples
of linkage functions include:

\begin{itemize}
\item \emph{Complete linkage} between two clusters is defined as the
  greatest distance of all possible pairs of items, one from each
  cluster.
\item \emph{Single linkage} is similar to complete, but the smallest
  distance is used.
\item \emph{Mean linkage} uses the mean where minimum or maximum
  are used above.
\end{itemize}

\subsection{Clustering Media Content}
\label{sec:clustering-media}

For media content, agglomerative clustering is a good match for two
practical reasons:

\begin{enumerate}
\item Content is continuously published, so there is potential
  to consider a ``real-time'' clustering system where new items
  arrive in their own cluster initially and are merged into an
  appropriate, established cluster when the next merge is performed.
  This is out of scope for this work, but is an interesting
  consideration if an organisation is to apply this to its real
  content in a way that keeps up to date with new content as it is
  published.
\item The BBC has of the order of tens of millions of pages and thus
  the content is likely to be very diverse with little to no overlap
  between the vast majority of it. The agglomerative model appears to
  fit better with the idea of merging together items where small
  islands of similar content are found and simply ignoring the large
  ``dark matter'' of pages that don't easily merge or for which we
  cannot easily extract good metadata (e.g. if the pages are very
  old).
\end{enumerate}

In this experiment, Jaccard distance\cite{witten2005data}
based on its suitability for binary, asymmetric data. It is argued
that the vast vocabulary offered by all potential semantic web
RDF triples will lead to very asymmetric binary values based on the
presence or non-presence of that triple.

The chosen linkage function was the complete linkage
on the basis that it ensures that any two content items
within a cluster to be considered related. This avoids the so-called
\emph{chaining phenomenon} where single linkage may create clusters
where two unrelated items are members because they are each related
to an item of chain of related items
therebetween.\cite{everitt2011hierarchical}

\section{RDF and Feature Extraction}
\label{sec:rdf-and-features}

The RDF graph is a powerful model
for metadata based on representing knowledge as a set of
subject-predicate-object \emph{triples}. The query language, SPARQL, gives us a
way to query the RDF graph structure using a declarative pattern and return a
set of all variable bindings that satisfy that pattern.

For example, the SPARQL query in Listings~\ref{lst:sparqlfoaf}
queries an RDF graph that contains contact information and returns the
names and email address of all ``Person'' entities therein.

Notably, Kiefer, Bernstein and Locher\cite{kiefer2008adding} proposed a novel
approach called SPARQL-ML -- an extension to the
SPARQL\cite{segaran2009programming} query language with new keywords to
facilitate both generating and applying models. This means that the system
capable of parsing and running standard queries must also run machine learning
algorithms.

Their work involved developing an extension to the SPARQL query
engine for \emph{Apache Jena}\footnote{https://jena.apache.org/} that integrates
with systems such as \emph{Weka}\footnote{http://www.cs.waikato.ac.nz/ml/weka/}.
A more suitable software application for enterprise use might focus solely on
converting RDF graphs into a neutral data structure that can plug into arbitrary
data mining algorithms.

\begin{lstlisting}[label=lst:sparqlfoaf,caption={Example SPARQL query for people's names and email addresses},language=sparql]
PREFIX foaf: <http://xmlns.com/foaf/0.1/>
SELECT ?name ?email
WHERE {
  ?person a foaf:Person.
  ?person foaf:name ?name.
  ?person foaf:mbox ?email.
}
\end{lstlisting}

If we consider an RDF graph, $g$, to be expressed as a set of triples:

\begin{displaymath}
  (s, p, o) \in g
\end{displaymath}

\noindent this query could then be
expressed as function $f: G \rightarrow (S \times S)$ where $G$ is the set of
all possible RDF graphs and $S$ is a set of all possible strings.
This allows the result of the
SPARQL query to be expressed as a set of all SELECT variable bindings that
satisfy the WHERE clause:

$$
q(g,n,e) = \exists p . (p, type, Person) \in g\ \land (p, name, n) \in g \land (p, mbox, e) \in g
$$

$$
g \in G \models f(g) = \{(n, e) \subseteq (S \times S) \: | \: q(g,n,e)\}
$$

This could be generalised to express a given feature set as
vector $(a_1, a_2, ..., a_n)$:

$$
g \in G \models (a_1, a_2, ..., a_n) \in f(g)
$$

\noindent and in the case where all $a_k \in f(g)$ are literal (e.g. string or
numeric) values, we can thus consider a given SPARQL query to be specific
function capable of feature extraction from any RDF graph into sets of
categorical or numeric features.

\begin{lstlisting}[label=lst:sparqlabout,caption={SPARQL query to determine what },language=sparql]
PREFIX rdf: <http://www.w3.org/1999/02/22-rdf-syntax-ns#>
SELECT ?topic
WHERE {
  ?article rdf:about ?topic .
}
\end{lstlisting}

This might allow a query that extracts a country's population, GDP, etc.
provide feature extraction for learning patterns in economics, for example.
However, this is limited to features derived from single-valued predicates
with literal-valued ranges. It is not clear how to formulate a query that
expresses whether or not a content item is about a given topic.

In the RDF
model, it would be more appropriate to use a query like that in
Listings~\ref{lst:sparqlabout} where for a given $?article$ identified by
URI, we can get a list of URIs identifying concepts which the article mentions.
Such a query might be expressed as function $f': G \rightarrow \mathcal P(U)$ where $U$ is
set of all URIs such that:

$$
g \in G \models f'(g, uri) = \{t \: | \: (uri, about, t) \in g\}
$$

An approach of generating attributes for a given resource was proposed by
Paulheim and F\"urnkranz\cite{paulheim2012unsupervised}. They defined specific
SPARQL queries and provided case study evidence for the effectiveness of
each strategy.

Their work focused on starting with relational-style data (e.g. from a
relational database) and using \emph{Linked Open Data} to identify entities
within literal values in those relations and generated attributes from
SPARQL queries over those entities.

For a large content-producer, there is a more general problem where many content
items do not have a relational representation and the content source is a body
of text or even a raw HTML page. However, the feature generation from Paulheim
and F\:urnkranz proves to be a promising strategy given we can acquire an RDF
graph model for content items in the first place.

\section{RDF in the enterprise}
\label{sec:linked-enterprise-data}

At the end of chapter~\ref{chp:intro}, there is some discussion of the
benefits of using semantics and linked data for any process that we wish
to apply across the entire corpus.

In this section, the goal is to strengthen that case and review other
previous work around using linked data in the enterprise. This is a more
general argument for any enterprise with data (e.g. internal information
about staff or business processes) and not just media organisations.

\subsection{Data Fragmentation in Organisations}

The first thing to highlight is why enterprises struggle to work with all
the data spread across the organisation. Where all interesting data is
contained in a single database, it is relatively simple to develop software
that extracts all that data in a single query and processes it in some way.

It is rare, however, for any large business to keep all data stored in
one location and in one format. Even fairly simple businesses will have
separate data stores for their employee data, for example, and all
user names and passwords for the corporate email system. Without much effort,
a business that implements or purchases two distinct systems for personnel
and email has already fragmented its data.

There may be no difficulty from this initially, but then a problem arises
where it is noted that employees, in this example, do not automatically have
their email accounts disabled when they leave the organisation. Smaller
businesses might simple employ administrative staff to keep databases in
sync and implement a standard process to handle the case of an employee
leaving the company.

In a large organisation, it may be more desirable to automate this. This then
steps into the field of \emph{Enterprise Integration}, discussed in the next
section.

In a media organisation, it might be argued that while all employee and
internal systems data is in bespoke systems, we can ensure all ``content''
is stored in a single system such as a \emph{content management system}. This
can be strained by at least three different pressures:

\begin{enumerate}
\item diversity of content;
\item technology refresh cycles; and
\item fragmentation of teams.
\end{enumerate}

Where an organisation produces multiple \emph{types} of content, it
may be the case that one system is optimised for writing and editing textual
articles (e.g. news) and another system is purchased or created to handle
video content. If these systems are built by different teams or, worse,
are procured independently as commercial offerings, there is little incentive
for the content to be compatible for generalising techniques such as search
or data mining.

This is the compounded where development teams decide to rebuild software
systems to keep up to date with changes in the industry, implement new features
in ways not made possible by the old system or because the old system is
becoming more expensive to maintain than it is to replace. This also applies
for buying in new systems because a contract was up for renewal and the fresh
procurement process chose a competitor's commercial offering which is
incompatible than that of the previous provider.

When this happens, the teams might choose to migrate all the old content to
the system, but they also might choose not to if the older system is still
running without issues. If it can continue to serve old content in a
"read-only" state without issues, then there may not be a strong business
case to migrate that old content as long as it continues to work.

The teams may also choose to archive the old content in a flat state not
intended for editing. This allows the old system to be decommissioned, but it
does leave older content in a format not easily queries and certainly not
compatible with how newer content is actively managed.

The final problem highlighted above -- that of fragmentation of teams in the
organisation -- particularly affects very large enterprises such as the BBC.
Any media company might find that content production teams work best when
given sufficient autonomy. A consequence of this is that multiple teams
might implement very similar solutions to the same problems, but without the
incentive to share nor unify their approaches. From this, we can ultimately
end up with multiple content management systems even for the same type of
content, all actively maintained.

A lot of this falls out of what was observed by
Conway\cite{conway1968committees} (sometimes as known as \emph{Conway's Law})
in that an organisation's architecture ends up reflecting its communication
structures. From this, we can assume that if two teams in an organisation
are not encouraged to communicate well and often, then they are less likely to
share software code and data systems.

\subsection{Enterprise Integration}

The field of \emph{Enterprise Integration} is the study and application
of overcoming the fragmentation issues described in the previous
section. It is concerned with creating interconnectivity between
distinct systems, exchanging data between them and dealing with
distributed systems within the enterprise.\cite{vernadat2003enterprise}

\subsection{Organisational Difficulties with Enterprise Integration}
\label{sec:ei-difficulties}

Hohpe and Woolf\cite{hohpe2002enterprise} described a comprehensive
set of software design and architecture patterns that apply to this
field, but almost every pattern implies creating bespoke data flow
software for any given integration.

This means for every new system that is purchased or created, a
whole set of integration efforts are potentially initiated to move any
data into or out of that system.

A significant barrier to creating holistic views of all data
produced by an organisation is that each integration brings in a
vertical slice of all data required, but only for once source at a
time. If a hypothetical organisation has six content management
systems with actively-published content within them and each
integration effort takes one month, then it is a minimum of six
months before it can create something like a search feature over all
of that content.

\subsection{Linked Enterprise Data}

Allemang\cite{allemang2010semantic} proposed that a
\emph{linked data enterprise} can be truly agile by removing bottlenecks
created by enterprise integration efforts. There is an appealing
argument in using controlled vocabularies and efforts in linking
those vocabularies between systems to join up existing data silos
rather than trying to write integration layers to move data between
those silos.

A more detailed approach was outlined by Servant\cite{servant2008linking}
in how linking enterprise data might work with the view that it
could be adopted within Renault.

Real research in this area still seems sparse in the academic
literature, but there is arguably a case to be made for these
approaches for an organisations public-facing content given that is
already available on the public web.

The concept of linked enterprise data could encourage media
organisations to consider using linked data within all internal APIs
and data stores where possible such that reasoning across all that
information is more readily enabled. In this research, it is demonstrated
that we can get some value at least from the semantics available via
the content as it surfaces on the public World Wide Web, but this
value can be only increased if content producers improve the quality
of those semantics or linked data principles are used within internal
systems to reduce the need for bespoke enterprise integration effort.

\chapter{System Design}
\label{chp:design}

In this chapter, a system is inductively derived and concretely design to make
use of multiple strategies for:

\begin{enumerate}
\item gathering (meta)data about all of an organisations content items;
\item extracting metadata not explicitly modelled in source content management
systems;
\item further enriching that metadata with information not explicitly present
in the content item itself; and
\item applying machine learning to that content metadata to gain new insights
about that content.
\end{enumerate}

Initially, uses cases and a brief technical architecture for the
proposed system are given and then a theoretical data pipeline is
inductively defined that is the core proposition that extracts
semantics and prepares them for machine learning.

It is assumed that an organisation is already publishing its content
on HTML pages available on the public Internet and that engineers
can make use of existing machine learning algorithms and libraries. It
is out of the scope for the paper to deviate from well-established
machine learning techniques such as hierarchical clustering.

\section{Use Cases}
\label{sec:usecases}

Figure~\ref{fig:use-case} describes how a proposed ``Content Clustering''
software application would be used by established roles in a media
organisation.

\begin{sidewaysfigure}
  \begin{center}
    \includegraphics[width=\linewidth]{diagrams/usecase.png}
  \end{center}
  \caption{Use case diagram for content clustering system\label{fig:use-case}}
\end{sidewaysfigure}

The two main classes of user are those who \emph{produce} content,
e.g. journalists and TV programme makers, and those who will then
benefit from a potential application of these clusters -- in this case,
a feature is proposed that allows website users to find content related
to content they have just finished. These are the people that
\emph{consume} the output of the system.

Note that content production breaks down into several, differing
categories. In figure~\ref{fig:use-case}, we can see two examples of
a journalist writing news articles and a TV production team
publishing information about a particular episode or series. This is
by no means exhaustive as we may well have games developers creating
education games for children, editorial staff publishing recipes that
have appeared on cooking programmes aired by the BBC or pages
delivering the weather forecast for a particular city.

The audience members also break down into multiple categories who
could be reading news articles, watching programmes on on-demand
catch-up TV (e.g. BBC iPlayer) or a child reading educational materials.
These categories are depicted to remind there are multiple ways to
consume content as well as produce it, but it is also important to
note these roles can be transient if we are to promote content truly
on its relevance irrespective of the data store that happens to hold
it.

For example, a child or adult reading informative or educational
material may be interested in a relevant documentary available on
TV catch-up or somebody reading a news article about a topic might
wish to be informed that there are more in-depth educational guides
describing that topic in greater details (e.g. a press release about
a science discovery leading to a full, informative guide on that
field of science).

\section{High-Level Architecture}

The use cases depicted and described in Section~\ref{sec:usecases} naturally
lead to an application with two distinct interfaces: notification of new
content to consider for data mining and retrieval of information about content
clusters created to date.

\begin{figure}[h]
  \begin{center}
    \includegraphics[width=\linewidth]{diagrams/component.png}
  \end{center}
  \caption{High-level component diagram with interfaces for each use case\label{fig:component}}
\end{figure}

Figure~\ref{fig:component} shows a high-level view of a ``Content Miner''
software application that provides both interfaces. A set of ``notifier''
applications can be created to connect different content production and
management systems to the data mining system such that it is notified when new
content is created. The notification need only be an \emph{IRI} that uniquely
identifies that content item.

A \emph{Content Promotion Module} is also depicted as the example application
of the use case where website visitors or audience members then use the
data mining system to find content related to a page they are currently viewing.

The content miner breaks down into two general components: a pipeline
that prepares the data for machine learning and the clustering at the
end on that prepared data. The primary focus of this work is to
evaluate approaches for the former in the form of a
\emph{data pipeline} that uses semantics to take in a content item's
identifier at the source end and produces feature sets for data mining
at the output end. This data pipeline is inductively defined and
designed in the following section.

\section{Data Pipeline}
\label{sec:design-pipeline}

As stated in the previous section, a core subsystem in the overall
system is a conceptual data pipeline whose input is a URI or IRI
identifying a content item published on an organisation's website and
the output is feature sets ready for applying machine learning.

In this section, a theoretical pipeline is inductively defined in
steps such that an application of this pipeline would choose to
implement some subset of all potential pipeline stages as appropriate
for the relevant problem domain.

In Chapter~\ref{chp:implementation}, a system is engineered that implements
as many of these pipeline stages as possible such that a running instance of
the application can configure which components to use and which not to use.
Then in Chapter~\ref{chp:evaluation}, an evaluation of the system is given
while it is running each component in isolation to demonstrate which of the
theoretically-defined processes in this chapter appears to be most effective
in generating feature sets specifically for clustering web content.

\subsection{Definitions}

This system requires some initial definition of some data structures in use:

\begin{description}

\item[IRI] \hfill \\
The input to the system is a character string conformant to the
IRI syntax defined in RFC 3987\footnote{http://tools.ietf.org/html/rfc3987}.
This allows more generality offered by
URIs\footnote{http://tools.ietf.org/html/rfc3986} but is trivially made
compatible with systems that use URIs through the conversion algorithm defined
in section~3.2 of RFC 3987. Note that the public URL by which the public can
read or otherwise consume the content is a valid identifier, but we are not
restricted to that.

\item[Feature Set] \hfill \\
The final output of this data pipeline is a data structure
analogous to a relation or tuple per IRI fed into the system. Every IRI should
have a literal value against all possible columns or fields. For binary fields,
(e.g. the presence of absence of a concept tag), a more pragmatic structure
might be a list of tags positively associated with the IRI rather than
explicitly assigning $false$ to all tags to which the content item does not
pertain. This is analogous to a spare matrix when dealing with a large number
of dimensions.

\item[Named RDF Graph] \hfill \\
The structure used throughout most of the data pipeline
is that of an RDF graph. This is used for all the benefits outlined in
Section~\ref{sec:linked-enterprise-data} such as ease of transformation and
combining of data sets. Named graphs are used such that all data acquired
are keyed back to the IRI of the content item being processed. This also allows
all graphs to be combined in a \emph{triplestore} if needed to allow SPARQL
queries across the combined data for all content items. This can be modelled as
a data structure in many programming languages, but where a serialisation is
used (e.g. examples shown here or to send the data between components), the
JSON-LD\cite{sporny2014json} syntax will be used.
\end{description}

\subsection{Identity Graph}
\label{sec:identity-graph}

\begin{centering}
\begin{lstlisting}[label=lst:jsonld-identity,caption={Identity graph for a content item in JSON-LD syntax},language=json]
{
  "@id": "http://example.com/entity/1",
  "@graph": []
}
\end{lstlisting}
\end{centering}

With the knowledge only of a content item's IRI, we are arguably only able to
produce an empty named RDF graph. Such a graph for an example IRI
\texttt{http://example.com/entity/} is illustrated in
JSON-LD syntax in Listings~\ref{lst:jsonld-identity}.

The most na\"ive feature set we can generate from such an RDF graph is clearly
a singleton relation \texttt{("http://example.com/entity/")} where a single
$IRI$ field has the value \texttt{"http://example.com/entity/"}. It is also
clear that a set of one-dimension feature vectors with unique values in each
is not suitable for any form of machine learning activity. This does, however,
illustrate a baseline for a working software application that is -- at least
in the syntactic sense -- transforming IRI inputs to feature sets outputs.
Such a $null$ feature generator is depicted in Figure~\ref{fig:gen-null}.

\begin{figure}[h]
  \begin{center}
    \begin{dot2tex}[dot,options=-t math,autosize,pgf,scale=0.7]
      digraph g {
        rankdir=LR;

        node [shape=circle,margin="0,0"];
        edge [len=2];

        IRI -> RDF [label="extract"];
        RDF -> Features [label="(IRI)"];
      }
    \end{dot2tex}
  \end{center}
  \caption{Null feature generator \label{fig:gen-null}}
\end{figure}

Note that Figure~\ref{fig:gen-null} shows all three data structures involved
despite having no functional use. We can also see top-level definitions of the
process where we first \emph{extract} semantic information in RDF from a
content item identified by IRI and then \emph{generate} features therefrom.
More useful models can now be inductively defined by adding atomic
subcomponents that may each add value to the overall transformation.

There are three clear axes along which we can improve this pipeline:
\emph{extract} more RDF data knowing only an item's IRI,
expand or \emph{enrich} an existing RDF graph
and then improve how we \emph{generate} features for data mining.
In the first instance, we can consider the former and add
a single pipeline stage for expanding the RDF graph.

\subsection{RDF Extraction}
\label{sec:rdf-extraction}

Tim Berners-Lee outlined four rules\cite{berners2011linked} for Linked Data,
rule number three of which states ``When someone looks up a URI, provide useful
information, using the standards''. If we assume that many pages have embedded
some semantic web or RDF data, then a simple extraction strategy would be
to deference the content item's IRI via an HTTP GET and pass the content
to a parser capable of extracting RDFa, microformats, etc.

Many tools such as the RDFLib\footnote{https://github.com/RDFLib/rdflib}
provide functionality for taking a URL and returning an RDF graph of all data
found when fetching the resource it represents, so this is arguably an ideal
first choice in attempting to learn something about a content item from its
IRI.

\begin{figure}[h]
  \begin{center}
    \begin{dot2tex}[dot,options=-t math,autosize,pgf,scale=0.7]
      digraph g {
        rankdir=LR;

        node [shape=circle,margin="0,0"]

        IRI -> RDFa [label="dereference"];
        RDFa -> RDF [label="parse"];
        RDF -> Features [label="(IRI)"];
      }
    \end{dot2tex}
  \end{center}
  \caption{Semantic web data extraction\label{fig:gen-rdfa}}
\end{figure}

Figure~\ref{fig:gen-rdfa} depicts the pipeline with a simple dereference
step added. Note that the feature set generated is still the singleton
relation with only the IRI value. Thus the next step should be to add a step
that improves the feature set generation.

\subsection{Feature Set Generation}

Paulheim and F\"urnkranz\cite{paulheim2012unsupervised} described a number of
SPARQL queries for generating feature sets from RDF data, which could inspire
a simple query such as that shown in Listings~\ref{lst:simple-sparql}. This
query generates a boolean \texttt{true} value for any properties that match
and implies \texttt{false} for those that do not.

\begin{lstlisting}[label=lst:simple-sparql,caption={Generates field \texttt{content\_?p\_?v} with value \texttt{true}},language=sparql]
SELECT ?p ?v
WHERE { ?iri ?p ?v . }
\end{lstlisting}

As also noted by Paulheim
and F\"urnkranz, this overlooks the
\emph{open world assumption}\cite{russel2010artificial}. However, application
of clustering algorithms on binary data can employ asymmetric distance metrics
such as Jaccard similarity coefficient\cite{witten2005data}, which notably
avoids deriving similarity from negative values. That is, two content items
lacking a particular property will contribute no information about their
(dis)similarity. Thus we safely avoid inadvertently grouping together one item
that genuinely lacks the property with another that indeed has the property,
but we lack the positive assertion thereof in the data extracted.

\begin{figure}[h]
  \begin{center}
    \begin{dot2tex}[dot,options=-t math,autosize,pgf,scale=0.7]
      digraph g {
        rankdir=LR;

        node [shape=circle,margin="0,0"]

        IRI -> RDFa [label="derefernce"];
        RDFa -> RDF [label="parse"];
        RDF -> Features [label="content\_?p\_?v"];
      }
    \end{dot2tex}
  \end{center}
  \caption{Semantic web content extraction with basic SPARQL feature generation\label{fig:gen-rdfa-basic}}
\end{figure}

The basic pipeline in Figure~\ref{fig:gen-rdfa} can thus be augmented with this
basic feature extraction to produce the pipeline depicted in
Figure~\ref{fig:gen-rdfa-basic}.

\subsection{RDF Enrichment}

The third and final direction in which this data pipeline can be improved is
in terms of data enrichment. A simple strategy here is to repeat the
dereferencing used in Section~\ref{sec:rdf-extraction}, but for each IRI
found as the object of a triple in which the initial IRI is the subject.
Formally:

$$
g \in G \models \exists p, o . (IRI, p, o) \in g \rightarrow g' = deref(o)
$$

That RDF graphs can be modelled as mathematical sets as in
Section~\ref{sec:rdf-and-features} means we can express a graph enriched this
way as a \emph{union} of the initial graph with each graph returned from all
dereferencing:

$$
g \in G \models g' = g \cup \bigcup \: \{deref(o) \: | \: (IRI, p, o) \in g\}
$$

Figure~\ref{fig:object-deref} shows the data pipeline with this additional
enrichment stage. Note that now we have potential for some stages being
executed in parallel.

\begin{sidewaysfigure}[h]
  \begin{center}
    \begin{dot2tex}[dot,options=-t math,autosize,pgf,scale=0.7]
      digraph g {
        rankdir=LR;

        node [shape=circle,margin="0,0"];

        dummy [shape=none,label=""];

        IRI [label="IRI"];
        RDF1 [label="RDF_1"];
        RDF2 [label="RDF_2"];
        RDFp [label="RDF'"];

        IRI -> RDFa [label="dereference"];
        RDFa -> RDF [label="parse"];
        RDF -> RDF1 [label="dereference/parse"];
        RDF -> RDF2 [label="dereference/parse"];
        RDF -> RDFp [label="\cup"];
        RDF1 -> RDFp [label="\cup"];
        RDF2 -> RDFp [label="\cup"];
        RDFp -> Features [label="content\_?p\_?v"];
      }
    \end{dot2tex}
  \end{center}
  \caption{\label{fig:object-deref}Semantic web content miner with additional dereferencing of linked entities}
\end{sidewaysfigure}

So far in this section, we have inductively built up a data pipeline from
a ``null'' base working with only identity graphs to a simple pipeline
capable of \emph{extracting} an RDF graph, \emph{enriching} it and then
\emph{generating} features from it.

What needs to be proven through experimentation is \emph{which} of these
provides the information required for effective data mining. As part of this
experimentation, we can now look at further techniques and approaches to try.

\subsection{Improving Extraction}

In order to gain a larger set of RDF data at the start of the pipeline, we
can derive further ways to get information about a content item given only
its IRI as input.

Rizzo and Troncy\cite{rizzo2012nerd} defined a framework called NERD capable
of combining multiple entity extraction systems to provide a unified way of
identifying -- and disambiguating -- named entities within a given body of text.
With such a system, we can create a second, parallel RDF extraction strategy
that creates a graph of triples in the form:

$$
(IRI, rdf\!\!:\!\!about, Entity)
$$

\noindent where $Entity$ is an IRI representing a concept or entity believed
to be found in the content's textual content. A data pipeline complementing
the RDFa-based extraction is depicted in Figure~\ref{fig:entity-extraction}.
Note the ability to apply a simple set union to the result of each extraction
as with the enrichment.

\begin{sidewaysfigure}[h]
  \begin{center}
    \begin{dot2tex}[dot,options=-t math,autosize,pgf,scale=0.8]
      digraph g {
        rankdir=LR;

        node [shape=circle,margin="0,0"]
        RDF1 [label="RDF_1"];
        RDF2 [label="RDF_2"];

        IRI -> RDFa [label="dereference"];
        IRI -> Text [label="GET"];
        RDFa -> RDF1 [label="parse"];
        Text -> RDF2 [label="entity extraction"];
        RDF1 -> RDF [label="\cup"];
        RDF2 -> RDF [label="\cup"];
        RDF -> Features [label="content\_?p\_?v"];
      }
    \end{dot2tex}
  \end{center}
  \caption{Named entity extraction in addition to semantic web extraction\label{fig:entity-extraction}}
\end{sidewaysfigure}

Another strategy can be to infer a relationship between two content items where
one contents an HTML link to another. It is not always possible to derive
precise semantics of such a link (unless the publisher has kindly provided
a \texttt{rel} attribute), but a weak relationship such as:

$$
(IRI_1, ex:related, IRI_2)
$$

\noindent might prove -- through experimentation -- to be useful enough
for data mining insights.

The fourth and final extraction strategy explored is acquiring metadata from
bespoke Content Management Systems and other internal APIs. This is generally
the only option used in enterprises settings, as discussed in
Section~\ref{sec:linked-enterprise-data}. This is likely to be the richest
source of information where an enterprise has typically preferred bespoke
integrations against non-hypermedia interfaces, so experimentation should
help quantity or qualify the value such a direct integration adds (perhaps to
consider it in combination with the cost of bespoke, repeated integration
projects).

The assertion explored here is that such bespoke integrations can
\emph{complement} cheaper work such as extracting RDFa with pre-built tools
(and thus be developed one-by-one after the initial release of an application
such as this data pipeline). Note that repeated custom integration projects
means that each data source requires a different application be developed (as
opposed to the reuse of a single RDFa parser or HTML link scraper). This
also means we are not necessarily comparing like-for-like if we introduce only
one at a time. It also makes it
difficult to evaluate data mining of a diverse content corpus if an integration
against an API provides additional metadata for only, say, 10\% of that corpus.

These challenges aside, it is clear that bespoke integrations have a clear place
in this data pipeline being applied in a real enterprise setting. Now that
we have a complete set of theoretical stages for \emph{extraction}, the
remaining improves lie now in the \emph{enrichment} and
\emph{feature generation} stages.

\subsection{Improving Enrichment}

In addition to enriching through dereferencing linked entities, it is proposed
to explore the following options:

\begin{itemize}
\item inferring relationships to hypernyms as defined by Wordnet\cite{miller1995wordnet};
\item inferring facts based using rules derived from expert domain knowledge;
\item using RDFS and OWL to generate new triples with well-established Ontology rules.
\end{itemize}

In the first approach, we can consider a relationship rule such as:

$$
(IRI, ex:related, ex:Dog)
$$

\noindent and \emph{infer} the fact:

$$
(IRI, ex:related, ex:Animal)
$$

\noindent and produce an enriched graph containing all additional facts
inferred in this way.

When dealing with proper nouns and named entities, inferring facts based on
domain knowledge may be more appropriate. For instance, the rule in n3 syntax:

\begin{lstlisting}
  {
    ?article ex:takesPlaceIn ?city .
    ?city a ex:City .
    ?city ex:capitalOf ?country .
  } -> { ?iri ex:takesPlaceIn ?country }
\end{lstlisting}

\noindent might be useful to help cluster articles that take place in the same
country -- even if the countries are not always explicitly mentioned therein.
Such an inference requires knowledge about cities and countries to
write and domain experts for different types of content might be able to offer
more nuanced rules.

An example for BBC content might be to infer that all articles written under
the \emph{Newsround} brand is suitable for children or that programmes that
have broadcast times during the day are also suitable for children.

The third and final proposed improvement makes use of standard tools to find
\emph{closures} using, e.g. RDFS, ontology rules. With this approach, we
can infer that entities that have a given type or class also have their
superclasses and supertypes. This gives us similar inference to hypernyms,
but with knowledge present in well-established ontologies.

An obvious example might where two content items have been identified as
related to the same concept -- so they would be candidates for clustering
together -- but when RDF data are extracted, it is found that two
different URIs have been used for each:

\begin{displaymath}
(IRI_{1}, <\!\!http\!:\!\!//dbpedia.org/property/related\!\!>, ex1:entity)\\
\end{displaymath}
\begin{displaymath}
(IRI_{2}, <\!\!http\!:\!\!//dbpedia.org/property/related\!\!>, ex2:anotherEntity)
\end{displaymath}

In the RDF graph for $IRI_2$, say, we might find the source had provided an
\texttt{owl:sameAs} assertion such as:

\begin{displaymath}
(ex2:anotherEntity, owl:sameAs, ex1:entity)
\end{displaymath}

This is possible in the case where the second item's data source uses its own
set of identifiers for entities, but has chosen to provide equivalences to
a more standard set of identifiers (e.g. DBpedia). With this information,
our data pipeline can infer:

\begin{displaymath}
(IRI_{1}, <\!\!http\!:\!\!//dbpedia.org/property/related\!\!>, ex1:entity)\\
\end{displaymath}
\begin{displaymath}
(IRI_{2}, <\!\!http\!:\!\!//dbpedia.org/property/related\!\!>, ex2:anotherEntity)\\
\end{displaymath}
\begin{displaymath}
(IRI_{2}, <\!\!http\!:\!\!//dbpedia.org/property/related\!\!>, ex1:entity)
\end{displaymath}

\noindent and the feature generation stage might provide the common features
\texttt{dbprop\_related\_ex1\_entity=true} for both content items.

\subsection{Improving Feature Generation}

The feature generation outlined so far relies solely on boolean values
indicating whether or not a given content item is related by some property to
some entity. This recreates the concept of \emph{tagging} where a given
object is either associated or not associated with a series of \emph{tags}.

One of the advantages of the RDF graph model is that we are not constrained
necessarily to properties and attributes directly applicable to the entity. We
could imagine adding a level of indirection to the query in
Listings~\ref{lst:simple-sparql} to create the query in
Listings~\ref{lst:level2-sparql}.


\begin{lstlisting}[label=lst:level2-sparql,caption={Generates field \texttt{content\_?p1\_?p2\_?v} with value \texttt{true}},language=sparql]
SELECT ?p1 ?p2 ?v
WHERE {
  ?iri ?p1 ?o .
  ?o ?p2 ?v .
}
\end{lstlisting}

With the this query, features of the form \texttt{content\_?p1\_?p2\_?v} can
be generated. An example of this might be where a television programme
content item has information about the actors that appeared therein, e.g.
\texttt{ex:hasActor}, and furthermore we have information about those actors
such as where they were born, e.g. \texttt{ex:bornIn}. With the path created
by following both of these predicates, it is possible to create features for
a television programme such as
\texttt{content\_ex:hasActor\_ex:bornIn\_Edinburgh} and we can potentially
find similarity between programmes where the actors were born in the same
city.

Perhaps a third step in the predicate path followed can give us even more
useful features. The example above could be expanded to
\texttt{content\_ex:hasActor\_ex:bornIn\_ex:cityIn\_Scotland} to allow the
more general ability to cluster programmes with Scottish actors, for instance.

Appropriate experimentation should show whether more value is gained by
adding these additional levels of indirection.

\subsection{Maximal Data Pipeline}

In this section, a data pipeline was inductively built up from a base,
identify pipeline with suggestions for potential improvements in different
stages. An application of all the ideas discussed so far might look like
that depicted in Figure~\ref{fig:maximal-pipeline}.

\begin{sidewaysfigure}[p]
  \begin{center}
    \begin{dot2tex}[dot,options=-t math,autosize,pgf,scale=0.8]
      digraph g {
        rankdir=LR;

        node [shape=circle,margin="0,0"]
        RDF1 [label="RDF_1"];
        RDF2 [label="RDF_2"];
        RDF3 [label="RDF_3"];
        RDFp [label="RDF'"];
        RDFp1 [label="RDF'_1"];
        RDFp2 [label="RDF'_2"];
        RDFp3 [label="RDF'_3"];
        RDFp4 [label="RDF'_4"];

        IRI -> RDF1 [label="\text{parse}"];
        IRI -> RDF2 [label="\text{extract}"];
        IRI -> RDF3 [label="\text{scrape}"];

        RDF1 -> RDF [label="\cup"];
        RDF2 -> RDF [label="\cup"];
        RDF3 -> RDF [label="\cup"];

        RDF -> RDFp1 [label="\text{dereference/parse}"];
        RDF -> RDFp2 [label="\text{hypernym inference}"];
        RDF -> RDFp3 [label="\text{expert inference}"];
        RDF -> RDFp4 [label="\text{OWL inference}"];

        RDF -> RDFp [label="\cup"];
        RDFp1 -> RDFp [label="\cup"];
        RDFp2 -> RDFp [label="\cup"];
        RDFp3 -> RDFp [label="\cup"];
        RDFp4 -> RDFp [label="\cup"];

        RDFp -> Features [label="\text{content\_?p\_?v}"];
        RDFp -> Features [label="\text{content\_?p1\_?p2\_?v}"];
      }
    \end{dot2tex}
  \end{center}
  \caption{\label{fig:maximal-pipeline}Maximal Data Pipeline}
\end{sidewaysfigure}

The goal of this work is to evaluate how the quality of unsupervised clustering
changes with respect to enabling different combinations of the maximal
data pipeline. In the next section, some of the technical architecture of this
pipeline is outlined and the next chapter will cover how the overall system is
then implemented.


\chapter{Implementation}
\label{chp:implementation}

In this chapter, some of the details of the system implementation
are described. Initially, we discuss some of the implementation
strategy employed due the experimental nature of the system and the
desire to run multiple, competing approaches side-by-side in an
efficient way.

The remaining sections walk through the implementation of each part
of the data pipeline from upstream to feature sets and the chapter
concludes with implementation of the clustering and generation of
clusters for evaluation.

\section{Software Architecture}

All functionality was implemented as part of a single Python module
named \emph{distillery}
that is run via the Unix command line, with different subcommands for
each feature. Each command was designed around the
Unix Philosophy\cite{raymond2003art} of doing one job per command and
that they were composable via Unix pipes. This allows pipelines to
be composed:

\begin{centering}
  \begin{lstlisting}[
      basicstyle=\scriptsize,
      label=lst:unix-pipe,
      language=bash]
distillery extract <iris.txt | distillery enrich | distillery generate
  \end{lstlisting}
\end{centering}

\noindent that resemble the entire theoretical data pipeline described
in section~\ref{sec:design-pipeline} without the need for middleware
such as message queues. More typical enterprise architectures around
message oriented middleware or event-driven architecture might need
to be employed to scale this system to millions of documents, but
the simple approach above is sufficient for tens of thousands of
documents.

Even this simple approach would scale very well with suitably low
network latency. The largest bottleneck observed was the necessity
to do at least one HTTP request per item ingested and then enrich
via an HTTP request per object of a triple where the IRI is a subject.
For some content items, this could easily be over 100 HTTP requests at
which point parallel CPU cores or asynchronous I/O programming does
not overcome the demands on network performance.

Throughput is a particular challenge when data processing systems
are first launched. The day-to-day volume of item creations and
updates may be low enough for even a low-scale application, but the
performance demands increase significantly when we wish to bootstrap
a backlog of all content items historically published -- of which
could easily be tens of millions for organisations such as the BBC.

For this, there is an
appeal to designing such a system on a cloud computing platform such
that it can be scaled up for the initial import of existing data and
then back down for ongoing updates when that import completes. Such
design is out of scope for this paper, but some discussion of
limitations of the existing system in section~\ref{} highlights where
this system may be architected differently to be more suitable for
such a cloud-based deployment.

The higher-level compositional architecture of the individual
data processing stages can, of course, be altered independently of
the design of those stages. The following sections will describe in
greater detail those composable modules.

\section{Avoiding Redundancy Across Multiple Experiments}

The design of the data pipeline in chapter~\ref{chp:design} would
suggest an implementation with each stage implemented as a function
converting a graph into a new, richer graph.

That is, whilst the different strategies illustrated in the maximal
data pipline in Figure~\ref{fig:maximal-pipeline} can be run in
parallel, the extraction stage itself simply converts the identity
graph described in section~\ref{sec:identity-graph} to a single graph
(after having taken the union of all the strategies).

In the experiment that is the subject of this paper, the intent was
to compare each of the extraction strategies individually, but also
all possible (union) combinations thereof. This presents at least
two potential implementation strategies:

\begin{enumerate}
\item Build the whole data pipeline and machine learning system such
  that it is configurable which strategies are invoked and run an
  instance per configurable combination; or
\item implement the software to produce all possible outcomes along
  the pipeline.
\end{enumerate}

The former approach is stronger if the aim is to build the system
closer to how it would be implemented in an enterprise: only one
set of clusters is needed and even a production, enterprise system
would be configurable if maintainers wished to enable or disable
features as desired.

The latter strategy is arguably better for experimentation as every
IRI fed into the source end of the pipeline is processed at the same
time for every extraction and enrichment approach. This gives better
assurance that the same input is used to evaluate each competing
system.

Other operational reasons include being able to run the system on a
single machine if necessary and that there is no repeated work:
extraction by one means can be used in isolation, but also feed into
a union with another technique without having to recalculate that
graph. That is, if we have graphs $A$ and $B$ generated once each,
then we can output $A$, $B$ and $A \cup B$ for comparison. A parallel copy
of each system would be calculating $A$ and $B$ twice each.

The limitation here is a production-ready implementation of the
system would have be built again from the ground-up, perhaps reusing
functions and library code from the experimental version, as the
application itself will be structured around this idea of multiple
outputs for each input. This is likely to be the case for any
experimental system and is arguably in line with Brooks' classic
assertion to ``Plan to Throw One Away''\cite{brooks1995mythical}.

\begin{centering}
  \begin{lstlisting}[
      label=lst:extract-all,
      language=python,
      caption={Python function that generates all possible RDF Graphs}]
def all_extracted_graphs(iri):
    
    extraction_strategies = [
        dereference, extract_entities, find_links
    ]

    # This creates all 3 RDF Graphs only once
    graphs = [s(iri) for s in extraction_strategies]

    # Empty set removed from powerset
    for idx, gs in enumerate(powerset(graphs) - {{},}):
        # union defined elsewhere: list(graph) -> graph
        graph = reduce(union, gs)
        # Loop counter idx tells us from which technique
        # combination each graph comes
        yield idx, graph
  \end{lstlisting}
\end{centering}

An illustrative Python function is shown in
Listings~\ref{lst:extract-all} where each of the three extraction
techniques is invoked only once and a powerset function is used to
generate all possible unions of those graphs and thus yield all
possible usages or non-usages of each strategy to generate RDF
graphs. The details of the three extraction functions are covered in
section~\ref{impl-extraction}.

\section{Obtaining IRIs}

It is a substantial undertaking to design, build and test a
production-quality enterprise system that processing all content
produced by a media organisation. An experimental system needs to
focus on quick results and therefore is optimised to work only over
a small sample of that content.

As introduced in section~\ref{intro-problems}, all that content is
also likely to be distributed across multiple databases, content
management systems and other stores. Whilst it is hypothesised in
this research that semantics and linked data provide a good way to
extract data about the total breadth of all content without any
bespoke integration against internal systems, there still remains
the enterprise integration problem of \emph{discovery} of those
content items.

Essentially, the data pipeline outlined in
section~\ref{sec:design-pipeline} notably requires that known IRIs of
content items are fed into it, but makes no statement about how those
IRIs are acquired. This is a deliberate design decision to decouple
the concepts of discovery from this extraction/enrichment workflow.
Thus item may easily be re-ingested several times if it is updated
(or simply periodically) and
a modular approach like this defends against upstream systems changing
and having to rebuild the triggers that notify when content is created
or updated.

What follows it that the enterprise integration task has been reduced
(we do not need to write bespoke code to extract basic data about
every content item) but not wholly eliminated (some custom adaptors
need to be created to notify the pipeline of creation or update events
in each respective data store).

However creation of any such notification adaptors is out of scope
for this research altogether, so a heuristic approach was employed
where ten different, themed content aggregation pages on the BBC
website\footnote{
  Including, among others, the BBC Homepage (http://www.bbc.co.uk),
  BBC Arts (http://www.bbc.co.uk/arts) and BBC Science
  (http://www.bbc.co.uk/science)
} were polled for any new content promoted by editorial teams.
This provided a way to find new items shortly after they are published
and also ensured the data sample ultimately used focused on content
that was deemed noteworthy or interesting in the last few months.

This is not an effective method for obtaining a large number of items
quickly as it is entirely dependent on the rate these aggregate pages
are updated. Running slowly over a long period yielded approximately
ten thousand individual IRIs with little effort nor need for large
amounts of integration work.

\section{Extracting RDF Graphs}
\label{sec:impl-extraction}

Listings~\ref{lst:extract-all} hinted at three individual functions
capable of obtaining an RDF graph for a given IRI, all based on the
extraction strategies outlined in section~\ref{sec:rdf-extraction}.

\subsection{Dereferencing}

Listings~\ref{lst:deref-simple} shows how trivial a dereference
function can be if we use the comprehensive RDFLib library for
Python. The library can handle content type
negotiation\cite{fielding2014hypertext} to ensure it can happily
dereference and parse any response that contains semantics,
particularly RDFa (including embedded Turtle) and microdata within an
HTML page.

\begin{centering}
  \begin{lstlisting}[
      label=lst:deref-simple,
      language=python,
      caption={Python function that generates all possible RDF Graphs}]
from rdflib import Graph
    
def dereference(iri):
    g = Graph(identifier=iri)
    g.parse(iri)
    return g
  \end{lstlisting}
\end{centering}

\subsection{Entity Extraction}
\label{sec:impl-entity-extraction}

DBPedia Spotlight\cite{isem2013daiber} was chosen to perform the
entity extraction. A hosted version is available as a free service and
it has been shown to be effective in many cases. A true comparative
study of the merits of alternative entity extractors is out of scope
for this paper.

\begin{centering}
  \begin{lstlisting}[
      label=lst:extract-entities,
      language=python,
      caption={Python function that uses DBPedia Spotlight to extract entities from web pages}]
import requests
from rdflib import Graph, URIRef, Namespace

FOAF = Namespace('http://xmlns.com/foaf/0.1/')
    
def extract_entities(iri):
    g = Graph(identifier=iri)

    text = requests.get(iri).text
    
    spotlight_response = requests.post(
        'http://spotlight.sztaki.hu:2222/rest/annotate',
        data={
          'text': text, 'confidence': 0.8, 'support': 20
        },
        headers={'Accept': 'application/json'},
        timeout=600)

    if spotlight_response.ok:
        r_json = spotlight_response.json()
        if 'Resources' in r_json:
            annotations = {
                resource['@URI']
                for resource in r_json['Resources']
                }
            if annotations:
                for entity in annotations:
                    g.add((
                        URIRef(iri),
                        FOAF.topic,
                        URIRef(entity)))
    return g
  \end{lstlisting}
\end{centering}

Listings~\ref{lst:extract-entities} shows a basic implementation of
a function that fetches a page and then feeds the content into a
request to DBPedia Spotlight. All entities thus found are then fed
in as objects to a series of \texttt{foaf:topic} triples.

There is much that can be tweaked in this implementation and the
final version used in the experiment relied on a ad hoc whitelist
of XPATH queries known to narrow down the content to the main article
content on BBC pages. This is because the implementation shown in
listings~\ref{lst:extract-entities} does nothing to prevent entities
being extracted from page
chrome such as navigation links and other cross-site concerns.

In a given organisation, such an approach might well be sufficient to
extract main article content from a controlled set of pages. In a
production system, it may be more appropriate to consider
readability systems such as that from Arc90.

\subsection{Hyperlink Relationships}

In listings~\ref{lst:extract-entities}, we see a simple Python
function that uses the Beautiful Soup
library\footnote{https://www.crummy.com/software/BeautifulSoup/} to
search across all links in a document and infer a ``related''
property on the assumption that pages like to other pages that are in
some way related.

\begin{centering}
  \begin{lstlisting}[
      label=lst:find-links,
      language=python,
      caption={Python function generates triples based on links to other pages}]
import requests
from rdflib import Graph, URIRef, Namespace

DBPROP = Namespace('http://dbpedia.org/property/')
    
def find_links(iri):
    g = Graph(identifier=iri)

    text = requests.get(iri).text
    doc = BeautifulSoup(text)

    # For all <a> tags that have an href attribute...
    for a_tag in doc.find_all('a', attrs={
        'href': re.compile('.+')
    }):
        # ... handle href values being relative links ...
        other_iri = urljoin(iri, a_tag.attrs['href']).strip()

        # ... infer triple if IRI is to another BBC page
        if bbc_url.match(url):
            g.add((
                URIRef(iri),
                URIRef(DBPROP.related),
                URIRef(other_iri),
            ))

    return g
  \end{lstlisting}
\end{centering}

This falls foul of similar issues to entity extraction as described
in section~\ref{sec:impl-entity-extraction} in that site-wide
navigation links and other supporting page elements may link to
very general pages (e.g. a persistent link to ``home'' on every page).

With appropriate feature selection, this may have no impact, but for
this experiment a simple heuristic was employed to narrow down to
the ``interesting'' part of BBC pages.

Another feature in listings~\ref{lst:find-links} is a restriction
to consider only links to other BBC pages. There may be a positive
or negative effect in addtionally noting external pages linked to from
content pages, but that is left to future research to consider.

\section{Enriching and not Enriching}

The combinatorial consequence of the usages and non-usages of the
three extraction techniques described so far is that the
experimental system developed is already comparing seven ($2^n$ less
the case where no extraction is employed).

In order to maintain focus in the experiment, only one of the
enrichment techniques described in chapter~\ref{design} was
implemented and its usage and non-usage merely doubled the
number of systems for comparison to fourteen.

This proved sufficient to give indicative results as to whether there
is value added through any enrichment, but future research could
certainly compare the merits of different enrichment approaches.

\begin{centering}
  \begin{lstlisting}[
      label=lst:enrich,
      language=python,
      caption={Python function that enriches a graph via deferencing objects}]
import rfc3987
    
def get_objects(graph):
    return {str(row.o) for row in graph.query('''
            SELECT DISTINCT ?o
            WHERE {
              ?iri ?p ?o .
            }
            ''', initBindings={'iri': graph.identifier})}

def enrich(graph):
    new_graph = copy(graph)

    for o in get_objects(graph):
        if rfc3987.match(o, rule='IRI'):
            new_graph.parse(iri2uri(o))

    return new_graph
  \end{lstlisting}
\end{centering}

Listings~\ref{lst:enrich} illustrates a simple function to
deference all objects of triples where the current IRI is a subject.
Note the convention to set the graph's own identifier to that of
the content item of interest.

Using SPARQL in this way opens up possibilities to perform multiple
queries to choose IRIs to dereference. For example, we may wish to
include semantics for objects reachable by following two predicates
on the RDF graph and consider much more indirect data. Evaluating
the utility of this is left for future work.

\section{Feature Set Generation}

The feature generation strategy employed was dervied from a distilled
portion of the approach introduced by Paulheim
and F\"urnkranz\cite{paulheim2012unsupervised}.

Listings~\ref{lst:generation} shows an illustrative, recursive
function that ``walks'' the RDF graph starting with the content
item's IRI as the initial subject.

\begin{centering}
  \begin{lstlisting}[
      label=lst:generation,
      language=python,
      caption={Python function that generates feature sets from RDF graphs}]
from rdflib Literal, URIRef
    
def generate(g, subject=None, depth=3, features=None, prefix=''):
    if depth <= 0:
        return
    if not features:
        features = {}
    if not subject:
        subject = g.identifier
    for p, o in g.predicate_objects(subject=subject):
        if type(o) == URIRef:
            new_prefix = prefix + clean(p) + '_'
            features[new_prefix + clean(o)] = True
            generate(g, depth=(depth - 1), features=features, prefix=new_prefix, subject=o)
        elif type(o) == Literal:
            features[clean(p)] = str(o)

    return features
  \end{lstlisting}
\end{centering}

The \texttt{clean} function simply provides some cosmetic cleaning
up of predicate IRIs to prefer CURIEs as they are more compact. A
typical example feature might then be
\texttt{foaf:topic\_dbpedia:Albert\_Einstein: true} where the leaf
object is an entity or
\texttt{ogp:title: "Gandhi: Reckless teenager to father of India"} if
the leaf object is a literal.

\section{Feature Selection}

A simple selection heuristic was employed, again inspired by the work
of Paulheim and F\"urnkranz\cite{paulheim2012unsupervised}, to go
some way to reduce potential noise in the subsequent machine learning
phase.

In this experiment, all features that have the same value or entirely
unique values were removed and values that occurred twice or more
for a given feature had to make up over 5\% of possible values for
that feature. That is, if a feature had over 95\% of its values being
unique or all the same, then it was rejected.

\section{Clustering Implementation}

Hierarchical, agglomerative clustering was implemented by implementation
of a \texttt{cluster} function that was merge the two ``nearest''
clusters on each invocation. This assumes that all content items not
yet in a cluster are implicitly in a singleton cluster and the ability
to merge one at a time allowed for control of when to cease merging.

The closeness of two potential, candidate clusters was calculated via
a the complete linkage function as chosen in section~\ref{sec:clustering-media}.
This made use of an implementation of the Jaccard distance as also
justified in that same section.

\section{Results Generation Strategy}

With all of the components described so far in place, an experiment
was run processing nearly 10,000 content items from the BBC website
with each going into one of fourteen combination of extraction
and enrichment approaches.

For each approach, hiearchical clustering was applied repeatedly
until the next merge would produce a cluster with cohesion below 0.5.
This is reasonably arbitrary as each technique will have different
concepts of cohesion, but it provided a strong basis on which to
evaluate each approach on whether its cohesion meaningfully predicts
whether a human would agree that cluster contains related items.

\chapter{Results and Analysis}

\section{Comparison of Clusters Produced}

\section{Strengths and Weaknesses of Different Approaches}

\subsection{Embedded Semantics Extraction}
\subsection{Entity Extraction}
\subsection{Hyperlink Relationships}
\subsection{Enrichment by Dereference}

\section{Recommendations for Production Use}

\chapter{Evaluation}
\label{chp:evaluation}

This chapter builds on the more immediate analysis in
chapter~\ref{chp:analysis} of the clusters produced by evaluating the
results across a group of people. The system designed and built is
also a subject of evaluation, particularly any flaws or potential
improvements that emerge from having run the software and reviewed its
results.

A higher-level
goal of this research was to perform some more qualitative evaluation
of those clusters by obtaining human feedback on a potential
application thereof. The application evaluated was a feature that would
take a page a user is currently reading and suggest related content
based on their being members of the same cluster. Mock-ups of such
a feature were generated and evaluated via a survey as described
in section~\ref{sec:survey}.

The remaining sections of this chapter then discuss some issues with
the design and difficulties arising during implementation
respectively.

\section{Qualitative Survey}
\label{sec:survey}

An online survey was produced where a user is shown a sample page
from the BBC website and for each cluster of which that page is
a member, a row of up to four ``suggestions'' was presented to users.

In this section, some details around the design and sampling criteria
for the survey are discussed initially, followed by analysis of the
results. The section concludes with further qualitative feedback and
observations that arose in survey responses.

\subsection{Survey Design}

An initial challenge for generating the survey was that it was
necessary to present to users a view starting from a particular
content item. This as effectively an inversion of the analysis
from chapter~\ref{chp:analysis} where a top-down overview of the
clusters was performed, with some drill-down into more depth in some
cases.

A reasonably arbitrary sample size of 20 content items was chosen
for which it was therefore necessary to choose 20 \emph{pages} that
were known to be members of one or more clusters. It was also
desirable to ensure that multiple clusters are evaluated in one
survey question, so items that were placed into clusters by multiple
instances of the data pipeline were preferable.

It was also important to ensure that the overall set of items
chosen provided coverage across all permutations of the data pipeline
and that some diversity of content was present (e.g. all items are
not simply news articles without any representation of TV and video
content).

A further additional property chosen for the sample set was that
a diverse range of cluster \emph{cohesion} values were present in the
set. This arguably allows the survey to present for evaluation
clusters that the machine learning applied itself believes to be poor
as well as ones it believes to be cohesive. This controls
for coincidences where only good clusters were chosen for one approach
and another had its least cohesive results chosen. It also has
potential not just to compare which approaches receive the most
relevance votes, but also to look for where those approaches' results
produce clusters whose cohesion measure correlate with anything
meaningful to humans.

The above sampling heuristics were applied programmatically by first
inverting the cluster data structures such that there is a known
list of all content IRIs across the whole results set and the list
of clusters of which they are members are keyed against those IRIs.

A composite uniqueness key was then created for each membership those
content IRIs have in the form:

$$
u(iri, cluster) = (approach, cohesion(cluster), category(iri))
$$

\noindent where $approach$ is a unique key referring to a particular
permutation of the data pipeline being evaluated and $category$ is
a rough, heuristic function that tries to assess which part of the BBC
website ``owns'' that content item, e.g. BBC News, iPlayer, Weather,
Sport. Each item would have a uniqueness key attached to it for each
cluster of which it is a member.

Content items were initially scored based on how many clusters of
which they were a member:

$$
score(iri) = 10 - |5 - \left\vert{\{c \in C \: | \: iri \in c\}}\right\vert|
$$

\noindent where $C$ is the set of all clusters produced. This gives
us a score of 10 when the number of clusters in which the item appears
is the ideal value of five. The justification for this optimum is
based on the lower end of the memory capacity limits suggested by
Miller\cite{miller1956magical} (Miller's Law).

The item is thus ``punished'' when it appears in only one cluster
(thus the question would only be testing one approach and has less
value as a question) or if the item appears in too many clusters
(asking people to compare between too many items might be
overwhelming and reduce the potential for people to answer
accurately). As this is only a heuristic, the sampling code still
chose some items nearer this extremes, thus an absolute cap was
implemented in the survey itself to ensure a random sample of seven
clusters was chosen for evaluation when items appeared in a number
greater than seven.

Starting with the highest scoring so far, the sampling algorithm
then updated each score based on the number of unseen unique keys
it would introduce to the final sample if it were chosen:

$$
score'(iri) = score(iri) \times \frac{15 - |keys(iri) \cap K|}{2}
$$

\noindent where $keys(iri)$ is all uniqueness keys for an IRI:

$$
keys(iri) = \bigcup_{c \in C} u(iri, c)
$$

\noindent and $K$ is the cumulative union of all keys so far
generated by invocations of of the $keys$ function so far, with its
initial value being the empty set.

Some of the constants used were tweaked based on what appeared to
generated a desirable, mixed set of sample results such that they are
a little better than arbitrary values. Once all IRIs were updated with
the new scores, the final sample of
results for evaluation was simply the content items with the twenty
highest scores.

\subsection{Survey Results}

The survey was taken by 30 respondents with three having taken it
following the think-aloud protocol\cite{lewis1982using} so that more
information can be gathered around the motivation for the responses.
Two of the think-aloud participants were experts in usability testing
and user experience design and the remaining participant a normal
website user.

Details of observations captured during the think-aloud responses
are discussed in section~\ref{sec:eval-obs}.

A key metric generated from the responses was the proportion of times
respondents selected a ``related'' set of items based on clusters
generated from a particular data pipeline variant. A table of the
responses is given in table~\ref{tbl:results}.

\begin{table}[h]
  \centering
  \caption{Success rate for each pipeline variant in generating related content suggestions}
  \label{tbl:results}
  \begin{tabular}{lll}
    & Not enriched & Enriched \\
    Deference                                     & 5\%          & 44\%     \\
    Entity Extraction                             & 17\%         & 37\%     \\
    Hyperlink Relationships                       & 8\%          & 14\%     \\
    Dereference and Entity Extraction             & 34\%         & 37\%     \\
    Entity Extraction and Hyperlink Relationships & 16\%         & 11\%     \\
    Dereference and Hyperlink Relationships       & 26\%         & 30\%     \\
    All three extractions                         & 23\%         & 24\%     
  \end{tabular}
\end{table}

A tabular form of the results is helpful for showing which techniques
performed the best and whether enrichment improved or worsened the
performance. However, it is difficult to derive absolute meaning
from numbers taken from qualitative feedback over such a small
sample.

\begin{sidewaysfigure}
  \begin{center}
    \includegraphics[width=\linewidth]{diagrams/results.png}
  \end{center}
  \caption{Visualisation of performance of each data pipeline variant\label{fig:results}}
\end{sidewaysfigure}


In figure~\ref{fig:results}, each pipeline variant is visualised
as a node in a directed graph. Note the nickname RDFa to represent
dereferencing and extracting embedded semantics despite RDFa being
only one of a few semantic markups extracted in this case. Also note
the convention to suffix the variant name with $+$ to indicate it
has been enriched. The nodes
on the far left represent each of the three extraction techniques
used in isolation.

The edges of the graph represent introducing one new technique and
re-running the experiment. For example, the edge from \emph{RDFa} to
\emph{RDFa+} shows what happens to the performance if we introduce
enrichment via dereference to data extracted also by dereference.
The edges from e.g. \emph{Entities} and \emph{Links} point to
the result of having used both of those techniques in data extraction
(joining the graphs via set union).

Where a percentage increases (significantly), we can infer with some
level of confidence that adding in a particular technique
\emph{improves} the performance of the overall system for choosing
clusters of related content. Conversely, where the percentage drops
a large amount, we can infer that introducing that technique has
worsened the performance.

In the particular results depicted, there are a mix of insignificant
and significant changes in performance when we alter one aspect of
the system. Insignificant results include:

\begin{itemize}
\item In all but one case, the enrichment via dereference had
  very little effect. Changes between 1\% and 5\% do not stand out
  as large jumps in behaviour, particular with the small sample size
  of people being likely to make the error margins quite large.
\item Extraction via entity extraction does not seem to improve nor
  degrade when adding in hyperlink relationships.
\item The combination of dereference and hyperlink relationships
  does not perform very differently when entity extraction is added.
\end{itemize}

Some of the most significant results appear to be:

\begin{itemize}
\item Extraction via dereference only improves substantially when
  improved in \emph{any} way. It benefits most significantly out of
  all variants from enrichment.
\item Entity extraction benefits somewhat being combined with
  embedded semantics obtained via dereference, certainly more than
  it benefits from being combined with hyperlink relationships.
\item Hyperlink relationships perhaps benefit a little from
  enrichment (but it might be bold to claim that), but more significantly
  we can see that extraction technique in performance when combined
  with either of the two other extraction approaches (or both).
\end{itemize}

The step change between embedded semantics in isolation and the
enriched version thereof is so large it prompts a brief return to
reviewing the clusters each variant produced.

If we refer back to table~\ref{tbl:cluster-counts}, we can see
another significant step change in that the unenriched variant
produced 5 (fairly large) clusters and that number drops to 249 when
enrichment is introduced.

Considering also that this extraction approach performs the most
poorly across all results, we begin to see support for the analysis
in section~\ref{sec:anal-deref} where it was observed that only very
basic metadata is found embedded in many BBC pages. This appears to
have created a sample of feature sets that have a very small number
of features each. Further observations in
section~\ref{sec:anal-deref} stated that
many of the semantics found had
common values (e.g. \texttt{ogp:site\_name="BBC"}). That many features
thus did not pass feature selection (since they did not appear
with enough value variation across the corpus) means that the few
semantic properties found are stripped down even further.

Much of the performance gain by altering extraction via dereference
might be explainable purely due to having added more data properties
with which the machine learning algorithms can work. This starts to
suggest a potential for a minimum feature set size below which
clustering processes will struggle.

This is especially supported by
the large performance increase from having simply enriched the data.
Given the enrichment in a lot of cases just added more higher order
properties describing the semantic properties themselves (e.g.
\texttt{md:item\_rdf:type\_rdf:List} appeared in some cases, which
tells us only that the \texttt{md:item} property found takes
list objects as values), we can safely conclude much of the
``enrichment'' has not added information about the content, but
just increased the triple count enough to make the clustering more
effective.

Given the noise from page structure and meta-semantics like RDFS and
OWL, it might be surprising that embedded semantics performed well
at all. The strength that appears to have had the most impact is that
the few times semantic properties are used in BBC pages, content
producers are including basic categorical information such as what
part of the BBC site we are on (e.g. \texttt{ogp:site\_name="BBC Sport"})
and in some cases which section within that site
(e.g. \texttt{article:section="Golf"}).

The basic categories seemed to be powerful enough to please many
respondents who frequently approved of related content suggestions on
the basis that the suggestions were from the same part as the BBC
website. This may be an artefact of framing the survey in an
artificial way as people validating the system by whether the pages
are grouped by well-understand categories (e.g. football news, comedy
TV, science articles) does not necessarily correlate with how those
same users would have been tempted to follow links to suggested items
as part of their own casual browsing of the website. This point
was also raised in some of the think-aloud discussions with user
experience experts, as discussed in section~\ref{sec:eval-obs}.

It seems reasonable to conclude that embedded semantics perform
well enough to produce clustering that appears intelligent on casual
inspect. Given that it is one of the simplest to implement (run
a standard RDFa and Microdata parser such as RDFLib in Python), there
is much to say for its cost-benefit ratio. It could be concluded this
is a strong starting point for a first iteration of a real implementation
of this experimental system.

Entity Extraction performed most strongly out of all extraction
techniques in isolation. This result is even more promising when we
consider that the parameters of the DBPedia Spotlight extraction used
were arbitrarily chosen, so there is certainly scope for further
work to find better performance with better tuning of the entity
extraction alone.

The favourable discussion of both embedded semantics and entity
extraction above is supported somewhat by two observations in the
results visualisation: their combination appears to perform best out
of all pairings of extraction techniques and adding hyperlink
relationships to that pair reduces the performance. The joint
occurrence of these two facts may not be coincidental and thus
support the short analysis in section~\ref{sec:anal-hyperlink} where
even brief review of the clusters provides little confidence in their
utility. Some potential ways to improve this extraction were given
at the end of section~\ref{sec:production-recommendations}, but from
this research alone we cannot conclude any benefit in extracting
relationships in this way until those improvements are trialled.

It is also important to note conclusions from the lack of significant
step changes and the most interesting one in these results is the lack
of support for enrichment via dereference improving performance at
all. This follows on from analysis in section~\ref{sec:anal-enrichment}
that suggests the use of dereference simply falls down on the same
drawbacks as dereference as an extraction technique. The only case
where enrichment makes a large difference is improving extraction via
dereference, but this could be explained more by the latter performing
so poorly in the first place, as discussed above.

In summary, there is much promise to consider starting with
embedded semantics where the system will do little more than learn
categorisation of content (in the case of BBC pages, that is) and
look to bring in entity extraction in a controlled way so that
the tuning of the extraction can be experimented with.

A production version of this system should aim to achieve an evolution
from finding the ``obvious'' distinctions from the content (i.e.
do we really need machine learning to tell us that BBC Sport articles
are related to each other, but distinct from BBC Food recipes?) to
finding more nuanced relationships based on entities and topics found
within the content if it is monitored and tuned well. This difference
between clear-cut differences and more topic-based relationships
turned out to be a key theme from the think-aloud evaluation and
consequent discussions with user experience experts. This is
discussed in section~\ref{sec:eval-obs}.

\subsection{Further Observations and Discussion}
\label{sec:eval-obs}

Further qualitative feedback was obtained in two ways: firstly by
adding to the end of the survey some open-ended questions to invite
some discussion:

\begin{itemize}
  \item Do you feel that suggested links to promoted content is something you would use in general? Feel free to expand on your answer.
  \item Do you have any general observations on where the suggestions performed poorly or where they performed well?
  \item Please write any other comments or observations below.
\end{itemize}

Approximately half of respondents answered the first two and only
three out of the 30 went on to give some additional comments. Many
of those were user experience experts who felt they had much to
contribute additional feedback.

The second phase of additional, qualitative feedback was notes taken
during the ``think-aloud'' survey responses as described in
the previous section.

In the think-aloud observations, non-expert responses were terser
than those from experts and it seemed as though conclusions were
reached more quickly about
whether suggested content was relevant or related. The impression was
that non-experts may act on some instinct that is hard to
describe, whereas experts in user design are more experienced in
some of the rationales and patterns they have observed across many
user testing sessions. This more instinct-driven approach for
non-experts is supported further from observations that thumbnail
images were cited on more than one occasion as a motivator for
determining content as being similar or not.

The expert think-aloud observations highlighted the key difference
between content being similar due to being in the same categorisation
groups versus being related by themes or topics, as discussed at the
end of the previous section. The opinions from these user experience
experts was that a topic-based similarity would be strongest for users
to consider clicking through to the suggested content and this was
vindicated on a number of occasions in non-expert responses two
(in one example a respondent was disappointed that many suggestions
offered next to a BBC News article were just ``more articles'' and
not ``articles on the same topic'').

The topic-based feedback gives more support for getting good quality
metadata \emph{about} the content in far more fined-grained and
nuanced ways than broad categories typically found in the embedded
semantics in BBC pages. Entity extraction looks to be a strong
candidate to support this if we cannot get more nuanced topic tags
into the embedded semantics. This case is supported well by the
analysis of the main responses in the last section.

The expert discussions raised some cases where ``hard'' categories
for the content were necessary, however. In some cases the clustering
processes had grouped content suitable for children with content not
explicitly marked as such. For a media organisation like the BBC, where
there is wide provision of content targeted at children, any
promotional feature on the website needs to be very cautious about
what it suggests. In the BBC, it is typical to create a strict
separation between child and non-child content and a general-purpose
data extraction pipeline has no automatic knowledge of this. In a
real setting, it is likely a production version of this system would
want to have some basic business rules to constrain its operation
where these strict requirements exist.

Another case for hard categorisation came from BBC Bitesize pages
with GCSE revision material for 14 to 16 year old British children.
There was some approval where it suggested revision material
for other subjects (relating GCSE French to GCSE English, for
example) and even when it appeared to suggest other content aimed
at a younger audience (e.g. BBC Newsround news articles aimed at children),
but it was raised whether it was appropriate to ``distract'' children
revising for GCSE exams with entertainment content. This highlights
more consideration for business rules.

It was also noted
that GCSE revision material was sometimes grouped with material for
even younger age groups (e.g. Key Stage 2 for 11 year olds) which
is unlikely to be helpful. This is another drawback of trying to
``learn'' relationships without prior knowledge of strict categorisation
of the UK National Curriculum structure. The BBC has however modelled
this as a semantic ontology\footnote{http://www.bbc.co.uk/ontologies/curriculum}
which may be promising for future iterations of this research where
we can incorporate more inference and enrichment based on
ontologies.

The final case where strict separation of content is needed is where
the system chose to cluster many British football teams together.
Semantically, the system was behaving very well in identifying that
all pages are published by BBC Sport and talk about a football team,
its fixtures and its results. However, with proper domain expertise,
a system should know not to suggest arbitrary, alternative football
teams to a user who is on the page for their favourite team. It is
unlikely they wish to read about teams they do not follow.

One last point raised in discussions with user experience exports was
how they believe that users on a given part of the BBC site may be
in different moods or have different goals depending on where they
are. For example, users reading information and educational guides
are believed to be more likely to be in a position to be suggested
other informational content even if it's only tangentially related
because their focus will just be on reading interesting things.
Conversely, people reading articles on BBC News might only wish to
be suggested articles on the same topic or developing story.

These different moods or goals seem similar to some of the search
and discovery modes proposed by Russell-Rose, Lamantia
and Burrell\cite{russell2011taxonomy}, where they try to give names
to different modes of behaviour exhibited by users when using search
or navigation systems to find information. It is not immediately clear
how we can infer these different modes or use cases based on
the semantics of different content items, but there could be some
consideration to how knowledge of user behaviour can feed in to
any machine learning processes.

There is much useful feedback and discussion raised just from
discussion with some user experience experts and also non-experts as
they evaluated the system's output in the survey. The open questions
at the end of the survey prompted a large number of comments, which
can be summarised as:

\begin{itemize}
\item Some liked where clear categorical grouping had occurred, e.g.
  news articles grouped with news articles, perhaps from the same
  category. Areas of the site where clear categories are lacking
  from the embedded semantics were mentioned as performing less well.
\item Others were less happy with suggestions being driven by
  broad categories only and felt they needed to relate by topic
  before they would follow the links.
\item There were a range of opinions as to whether the results
  were overall good or bad.
\item Two or three comments supported the idea that more weakly
  related content was better on the information sites, also supporting
  the idea that users are in a more exploratory ``mode''.
\item At least one comment noted that content was old. The clustering
  has no knowledge of what is current and what is older (which is
  very important for things like news).
\item Comments generally supported the idea of having a related
  content feature at all, but many were quick to add that it had
  to be truly relevant otherwise they may ignore it in time or treat
  it as if it were spam or advertising.
\item There were many comments, even from BBC employees about
  having discovered parts of the BBC site they were not previously
  aware of. Some of this was helped by the sampling heuristic
  deliberately choosing a diverse range of content for evaluation, but
  it does support the idea that users might need features driven by
  data in this way to suggest parts of the website they did not
  know about before.
\end{itemize}

A common theme from both this section and the previous section is
this balance between structured, categorical data and topic and theme
information both feeding into how content is grouped. This provides
strong evidence again for the benefits of combining structured information
from embedded semantics with fuzzier approaches like entity extraction.

The simple survey designed and given to only 30 people has raised
a substantial amount of feedback on the variants of the system. In
the next section, there is some evaluation of the software itself
from a design perspective and some discussion of some of its
shortcomings that a production version might wish to avoid.

\section{Design Limitations}

Throughout chapter~\ref{chp:design}, the design of the data pipeline
and overall software architecture was outlined and known trade-offs
and limitations were highlighted. Further design trade-offs due to
the systems intended use to be experimental rather than
production-quality in an enterprise were also covered.

This section highlighted some design issues that were noted only after
the system was built and how research and industry alike might learn
from these problems. Such problems might be addressed if the results
of this experiment are to be researched further or indeed if such
a system were to be built in a real media organisation. Such
learnings for production use should follow on from the
recommendations in section~\ref{sec:production-recommendations} around
applying the data extraction techniques used and provide more general
software design recommendations.

\subsection{Performance}
\label{sec:eval-perf}

The experimental system ran far slower than would be acceptable in
an enterprise system and even to the extent that it would limit
developing this research to the scale where it is processing millions
of items -- a target that would be ideal for further proving the
suitability of semantics and machine learning on real media
content corpora.

A better-performing system might have allowed even this experiment
to consider larger numbers of items, although even a relatively
small sample size should sufficiently indicate which techniques look
more promising for developing a more efficient system that focuses
thereon (and save time by dismissing techniques that show no promise).

The performance difficulties that arose from the design and concept
broadly break down into two problems:

\begin{enumerate}
\item lack of consideration for data structures for the feature sets
  and clusters; and
\item network latency bottlenecks.
\end{enumerate}

The first problem stems from the fact that the design was precise
around the in-flight data in the pipeline, less consideration was
given for how to store the resulting feature sets and also the
clusters. The feature sets were stored as simple JSON objects on
files on disk, which is perfectly fast for writing and reading, but
there was no pre-calculated distance matrix between those objects. The
clusters were also stored in a na\"ive fashion of larger JSON objects
that embedded their members' data for quick display of those clusters
and calculation of things like cohesion.

A distance matrix is a $O(n^2)$ operation in terms of computational
complexity, which scales poorly, particularly for a continuous ingest
of data where each new item invalidates the matrix. The design of
the system implemented in this research was built around an assumption
that we wish to have a continuously-running data pipeline that can
update as new content is published. Recalculating the distance matrix
on each new item would effectively push the complexity up to $O(n^3)$
which is clearly undesirable.

With this in mind, a cut-off for content ingest had to be determined
so that the distance calculations and then the clustering could be
invoked. The system implemented does not save any intermediate data
such as distances, so adding in a single new item or cleaning out
any duplicates found required all the results to be regenerated from
the start.

A production version of this system would have to look at a distance
matrix and cluster set structure that allows for real time updates
of new and updated content, efficiently changing only parts of the
matrix affected (e.g. a new item only needs to calculate its distances
with each existing item and does not affect existing pairs). This
could be a promising direction for further research.

The other issue of network latency became very apparent in particular
when enrichment via object dereferencing was enabled. Whereas an
enterprise integration effort between two different data stores
operates on bespoke software engineered in a controlled network,
extracting RDF data via a URL dereference is susceptible to network
latency over the public Internet. Also, HTML pages with embedded
semantics have a very poor ratio of semantic metadata to irrelevant
information -- a document DOM focuses a lot on marking up the content
for web browsers.

There is still a compelling case for the \emph{decoupling} achieved
by using standards like RDFa and the \emph{abstraction} HTML and HTTP
provide over the underlying data stores, reducing the need for
bespoke clients. However, any extraction of that data needs to
tolerate or factor in the increased latency for this approach.

For extraction, the latency of a single HTTP request (or two or three
if HTTP redirects are issued) will only cause a bottleneck if new
content is published or updated at a rate faster than the pipeline
can process. This seems unlikely even in an organisation like the BBC,
particularly if the system implements structures such as queues to
handle bursts of input, processing them in quieter intervals.

However, for enrichment via deference, an RDF graph produced might
have hundreds of objects needing dereferenced via HTTP request. A
simple cache layer might prevent the system repeatedly requesting for
the same URL in a short period, but this kind of latency still means
it could take of the order of minutes for a single item to get
through the pipeline.

It is not clear that there is a single answer to handle this as it is
likely a combination of factors need optimised to make this form
of enrichment acceptable:

\begin{itemize}
\item It might be that hundreds of objects linked via predicates
  creates too much noise in general. If those objects have come mainly
  from entity extraction, should entity extraction be tuned to bring
  back only the ``main'' topics of a page?
\item Chapter~\ref{chp:analysis} discussed how many of the extraction
  techniques (particularly deference) produced a lot of irrelevant
  facts (e.g. about the page itself, not the content) all of which
  would be candidates for deference of the objects are IRIs.
  Effectively filtering out such noise would reduce the work spent
  enriching irrelevant things further.
\item A simple architectural solution is to parallelise the HTTP
  calls as much as possible such that an item's latency is only as
  long as the slowest response time. There are obvious limits to this
  (we should avoid overloading the server even if it is scaled for
  millions of visitors per day), but any parallelism is clearly
  better than a hundred sequential HTTP requests. Parallelism includes
  distributed architecture over multiple servers as much as
  concurrent programming paradigms.
\item An organisation should probably ensure a system is physically
  near any servers serving HTTP requests to reduce the overall TCP
  packet latency. This optimisation will add up over time, but is
  a challenge for a large organisation such as the BBC with systems
  of varying ages distributed over multiple locations
  \footnote{Industry has
  long outgrown the notion
  that enterprises like the BBC have a single ``www'' server as
  in the early days of the World Wide Web. However, the www hostname
  still needs to resolve to a single set of IP addresses for something
  that fronts the rest of the network like a load balancer. Effective
  caching in this front layer and ensuring our data pipeline lives close
  to that cache might be sufficient in this case. An example of a
  front cache might be something like the Varnish HTTP cache:
  https://www.varnish-cache.org/}
\end{itemize}

Even with all the optimisations possible, it is inescapable that
multiple HTTP requests to assemble an RDF graph is slower than a
single query to a data store's query engine (e.g. SQL) that returns
all the information needed in a single response. The benefits of using
semantics, HTTP and other standards -- such as decoupling, simple
enrichment, reuse of standard technologies -- need to be effectively
researched and promoted, otherwise enterprise organisations will
continue to reach for the seemingly obvious solution of building
software that queries one database to push to another.

\subsection{Noise}

The semantic web and more recently Linked Data are arguably built
around the idea that they embrace the loose architecture of the
World Wide Web. With that looseness -- open world assumption,
no schema constraints, etc. -- we cannot assure strict quality and
integrity of the data as would be possible in a closed-world
database.

The data pipeline design in chapter~\ref{chp:design} has no
requirements nor consideration for how to handle facts that are
erroneous, missing nor meaningless (e.g. see the discussion
in section~\ref{sec:anal-deref} of the system finding facts about the
pages' structure rather than content metadata).

In section~\ref{sec:impact-of-noise}, it is argued that noise might
not have as much impact as one would assume given the fuzzy nature
of machine learning in general. This argument is important to consider,
but here we explore the issue of noise on the assumption that it
\emph{is} a problem as it is important to evaluate how noise fed into
the system in the first place and how we might tackle it.

Erroneous facts may be non-trivial to distinguish from real facts, but
there is scope for research on effective methods for reducing that
noise. In an enterprise, we need not exclude the benefits of
enterprise integration development to \emph{support} this approach
based on semantics and web standards.

While there is much gained
from being able to reason about the entire breadth of an organisation's
content using semantics, it does not preclude writing bespoke business
rules that strip out known, erroneous data or provides a framework
against which to evaluate the likelihood of a fact being useful or
true. At the very least, there could be some work to develop ontologies
and schemas that detect inconsistencies in the data, but in a way
that adds value to a working system rather than such modelling being a
barrier to building the first iteration of the system.

Facts that are missing are arguably not a ``noise'' issue, but still
a data integrity concern. If an item lacks key information, it may
be categorised incorrectly. It should be noted that the data pipeline
designed here did nothing to address this deliberately and the desire
was to evaluate the open world characteristics of semantics without
trying to fill in the gaps.

As with the erroneous facts discussion above, there is much case to
be made for designing such a system so that it is extensible enough
with customisations and bespoke software to aid where it appears to be
struggling -- once its initial behaviour is evaluated. In the case
of the data pipeline designed in chapter~\ref{chp:design}, any
in-house databases and APIs can be integrated as additional extraction
techniques to be joined via union to more general approaches such
as dereferencing.

Data being irrelevant might be the larger of the three concerns here
as that greedy machine learning algorithms can get stuck on such
noise. In section~\ref{sec:few-large}, we see how one approach
caused the clustering to get stuck on the fact that all items linked
to a common FAQ page.

Other data attributes such as the Twitter
Cards metadata (see section~\ref{sec:smallest-cluster}) can contain
very broad things that are declared on all pages that use this
vocabulary (e.g. \texttt{card=summary} is always used to indicate that
a given page is using these meta tags) might cause a greedy algorithm
to focus too long on grouping together all pages that happen to choose
to implement this Twitter feature with less concern about the more
interesting properties.


\chapter{Conclusions and Future Work}

This chapter lays out some conclusions from this research around the potential
for semantics being used more in the enterprise, whether semantics combined
with machine learning or data mining has potential in industry and how
organisations should consider implementing this. Finally, some options and
potential for future research are suggested.

\section{Challenges and Opportunitues for Semantics in the Enterprise}

As discussed in section~\ref{sec:production-recommendations} and also more
generally throughout chapter~\ref{chp:analysis}, there is much scope for
enterprises to benefit from using semantics in public-facing pages that display
their own content. This is particulary necessary in larger organisations where
attemping to gain an overview of all the content metadata in a short timeframe
would be a large enterprise integration effort.

The challenges amount to the issues raised around noise (semantic information
not immediately relevant to the content), incomplete information (due to
the open world assumption) and perhaps even erroneous information (embedded
Semantic Web data is not always visible enough for publishers to spot errors.)

Even with these issues, the experiment performed in this research was still
able to take place across a wide range of BBC content without any bespoke
integration against internal data APIs nor a any significant code created
to deal with a particular data source.

The power to use semantics and other web standards to gain a broad overview
of a large set of content should be very appealing to a media organisation,
even if there are issues with that data, assuming their application is in
a use case where less precision of that information is tolerable.

Even where data needed corrected, the software developed even in this
experiment was able to adapt to broad patterns in the data that were
problematic. there is also much indication that engineers in the enterprise
would be to complement the breadth from extracting semantics with the depth
of data obtained directly from data sources. This is achievable with the
set-theoretic nature of RDF graphs and our ability to perform simple set
union operations to join data sources.

As is evidenced by the pressure from social media and networks such as
Facebook and Twitter, publishers will include better semantics if the right
incentives are there. Perhaps the incentives promised by the earlier semantic
web advocates (search engine optimisation, the development of clients that
join up data between websites, etc.) were not strong enough for publishers,
but the need to control how that content is summarised and discovered in
places outside of the primary website is indeed a strong incentive.

There is much to be said for the argument that any enterprise that therefore
creates applied features to make use of content metadata (promoting related
content being just one example), that the organisation might create appropriate
pressure internally for those publishers to "correct" or at least improve
their semantic annotations when faced with clear evidence where improvements
could be made and errors corrected.

\section{Suitability for Semantics and Data Mining in Media Organisations}



\section{Suggestions for Initial Implementations in the Enterprise}

\section{Future Research}
\subsection{Better Enrichment}
\subsection{Patterns for Reducing Data Noise}
\subsection{Deeper Study of Machine Learning on Semantics}



\bibliographystyle{plain}
\bibliography{references}

\appendix

\chapter{Proposal}
\includepdf{../proposal/proposal.pdf}
\chapter{Ethics Form}
\includepdf[pages=1-4]{ethics.pdf}
\chapter{Summary}
\includepdf{../paper/paper.pdf}

\end{document}
